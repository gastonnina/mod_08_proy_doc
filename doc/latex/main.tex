\documentclass[11pt,oneside,letterpaper]{book}

% ================= PREÁMBULO =================
\usepackage[utf8]{inputenc}
\usepackage[T1]{fontenc}
\usepackage[spanish]{babel}
\usepackage[a4paper, left=3.5cm, right=2.5cm, top=2.5cm, bottom=2.5cm]{geometry}
\usepackage{graphicx}
\usepackage{fancyhdr}
\usepackage{setspace}
\usepackage{titlesec}
\usepackage{tocloft}    % Para personalizar el índice
\usepackage{lipsum}
\usepackage{subcaption}
\usepackage{tabularx}
\usepackage{ragged2e}
\usepackage{array}
\usepackage{booktabs} % para tablas
\renewcommand{\arraystretch}{1.5}
%\usepackage{url}

% Cambiar fuente y tamaño del título
\usepackage{etoolbox}
\patchcmd{\lstlistoflistings}
  {\section*{\lstlistlistingname}}  % lo que hace por defecto
  {\section*{\centering\large\bfseries \lstlistlistingname}} % lo que tú quieres
  {}{}

%\usepackage{hyperref}
%\usepackage[hidelinks]{hyperref}  % Índice con links ocultos para el PDF final
%\usepackage{xurl} % Mejora cortes automáticos en URLs largas
\usepackage[
  colorlinks=true,
  linkcolor=black,
  citecolor=black,
  urlcolor=blue
]{hyperref}

\usepackage{listings}
\usepackage{xcolor}

\lstset{
  basicstyle=\ttfamily\footnotesize,
  keywordstyle=\color{blue}\bfseries,
  commentstyle=\color{gray}\itshape,
  stringstyle=\color{orange},
  breaklines=true,
  frame=single,
  columns=flexible,
  keepspaces=true,
}
\lstdefinestyle{pythonStyle}{
  language=Python,
  basicstyle=\ttfamily\small,
  keywordstyle=\color{blue}\bfseries,
  stringstyle=\color{orange},
  commentstyle=\color{gray}\itshape,
  numbers=left,
  numberstyle=\tiny,
  stepnumber=1,
  numbersep=5pt,
  backgroundcolor=\color{white},
  showstringspaces=false,
  breaklines=true,
  frame=single,
  captionpos=b
}

\lstset{style=pythonStyle}
\renewcommand{\appendixname}{Anexo}

\usepackage{float}
\usepackage{amsmath,amssymb}

\usepackage{mathptmx} % Usa Times New Roman como fuente principal

\usepackage[most]{tcolorbox} % cuadros de colores

% Bibliografía
\usepackage{csquotes} % Recomendado por biblatex
\usepackage[style=apa, backend=biber, citestyle=apa, language=spanish, url=true, doi=true, eprint=false]{biblatex}
\addbibresource{referencias.bib}
% Forzar formato de URL y DOI como clickeables
%\DeclareFieldFormat{url}{\url{#1}}
%\DeclareFieldFormat{doi}{\href{https://doi.org/#1}{https://doi.org/#1}}

% Márgenes y espaciado según APA
% APA: 1 pulgada = 25.4 mm (puedes dejar left=30mm si tu uni lo exige, si no, usa 25mm)
\geometry{
    letterpaper,
    left=25mm,     % Cambia a 30mm solo si lo pide tu universidad para empastado
    right=25mm,
    top=25mm,
    bottom=25mm,
    headsep=10mm   % Espacio entre cabecera (paginación) y el texto
}
\setlength{\headheight}{15pt} % Altura del encabezado
\setlength{\parindent}{1.5em}
\setlength{\parskip}{1em}
\onehalfspacing

\hyphenpenalty=10000
\exhyphenpenalty=10000

% Ruta para imágenes
\graphicspath{{images/}}

% Mostrar "CAPÍTULO" en mayúsculas y números romanos para capítulos
\renewcommand{\chaptername}{CAPÍTULO}
\renewcommand{\thechapter}{\Roman{chapter}}

% Secciones numeradas como 1.1, 1.2, etc.
\renewcommand{\thesection}{\arabic{chapter}.\arabic{section}}

% Formato visual del título del capítulo
\titleformat{\chapter}[block]
  {\normalfont\bfseries\Large}
  {CAPÍTULO \thechapter\-}{1em}{}

% ========= Personalización del índice =========
\renewcommand{\cftchappresnum}{CAPÍTULO }
\renewcommand{\cftchapaftersnum}{\quad}
\renewcommand{\cftchapnumwidth}{8em}

% ========= Encabezado APA =========
\pagestyle{fancy}
\fancyhf{}
\fancyhead[R]{\thepage} % Número de página en la parte superior derecha
\renewcommand{\headrulewidth}{0pt} % Sin línea en el encabezado
\renewcommand{\footrulewidth}{0pt} % Sin línea en el pie

% Espaciado visual del título de capítulo
\titlespacing*{\chapter}{0pt}{-30pt}{30pt} % Puedes ajustar -20pt a 0pt o mayor si el capítulo queda muy arriba

% enumerar indice hasta 3 nivel
\setcounter{secnumdepth}{3}
\setcounter{tocdepth}{3}

% justificar
\sloppy

\renewcommand{\thechapter}{\arabic{chapter}}

% ================= DOCUMENTO =================
\begin{document}

% CARÁTULA
\pagenumbering{gobble} % quita numeracion
\newpage
\thispagestyle{empty}
\begin{center}
	\Large{\textbf{UNIVERSIDAD MAYOR DE SAN ANDRÉS}} \\
	\large{\textbf{FACULTAD DE CIENCIAS PURAS Y NATURALES}} \\
	\large{\textbf{POSTGRADO EN INFORMÁTICA}} \\
	\vspace{0.5cm}
	\large{\textbf{MAESTRÍA EN INTELIGENCIA ARTIFICIAL Y DATA SCIENCE PARA LA TRANSFORMACIÓN DE NEGOCIOS }} \\
	\vspace{0.5cm}
	\large{\textbf{MODALIDAD VIRTUAL}} \\
	\large{\textbf{GESTIÓN 2024 - 2025}} \\
	\begin{center}
		\includegraphics[width=4cm]{00_logo_umsa.png}
	\end{center}
	\vspace{0.5cm}
	\large{\textbf{MODELAMIENTO DE DATOS I}} \\
	\vspace{1.2cm}
	\large{\textbf{
			MODELO SUPERVISADO PARA PREDECIR LA ADOPCIÓN DE IA
		}} \\
	\vspace{0.5cm}
	\textbf{POR:} LIC. GASTON HUMBERTO NINA SOSSA \\
	\textbf{DOCENTE:} M. SC. RODRIGO HUGO AVALOS QUISPE \\
	\vspace{0.5cm}
	LA PAZ – BOLIVIA \\
	Noviembre, 2025
\end{center}

\pagenumbering{roman} % ← Numeración romana (i, ii, iii)
\newpage
\section*{Dedicatoria}
\vspace{4cm}

A mi esposa Lis, por su amor, paciencia y apoyo incondicional.\\

A mis hijos Zoe e Ian, fuente inagotable de motivación y alegría.\\

Este logro también es de ustedes.


\newpage
\begin{center}
  \textbf{Resumen Ejecutivo}
\end{center}

\noindent El presente informe desarrolla un modelo supervisado para predecir la adopción de herramientas de Inteligencia Artificial por desarrolladores profesionales, utilizando el dataset público \textit{Stack Overflow Annual Developer Survey 2025}. El estudio aplica técnicas de análisis exploratorio, modelamiento y validación utilizando regresión logística y Random Forest, evaluados mediante F1-score con validación cruzada k-fold.

Los resultados muestran que la regresión logística sin balance alcanza el mejor equilibrio entre rendimiento, estabilidad e interpretabilidad (F1=0.8817). Se presentan además limitaciones, implicaciones prácticas y recomendaciones para organizaciones tecnológicas en procesos de adopción de IA.

\bigskip
\noindent\textit{Palabras clave:} clasificación supervisada, adopción de IA, regresión logística, random forest, validación cruzada, Stack Overflow Developer Survey.

\newpage
\fancypagestyle{plain}{
	\fancyhf{}
	\fancyhead[R]{\thepage}
	\renewcommand{\headrulewidth}{0pt}
}
\renewcommand{\cftchapdotsep}{\cftdotsep} % activa puntos suspensivos para capítulos
\renewcommand{\contentsname}{ÍNDICE}
% Cambia el tamaño del título del índice
\renewcommand{\cfttoctitlefont}{\large\bfseries}  % o \Large, \normalsize, etc.
\renewcommand{\cftloftitlefont}{\large\bfseries}
\renewcommand{\cftlottitlefont}{\large\bfseries}

\tableofcontents

\cleardoublepage   % O \newpage
\phantomsection    % Asegura que se enlace bien si usas hyperref
%\addcontentsline{toc}{chapter}{Índice de Figuras}  % Agrega entrada al índice general
\renewcommand{\listfigurename}{\MakeUppercase{Índice de Figuras}}
\listoffigures


\cleardoublepage   % O \newpage
\phantomsection    % Asegura que se enlace bien si usas hyperref
\renewcommand{\tablename}{Tabla}
\renewcommand{\listtablename}{\MakeUppercase{Índice de Tablas}}
\listoftables

% \cleardoublepage   % O \newpage
% \phantomsection    % Asegura que se enlace bien si usas hyperref
% \renewcommand{\lstlistingname}{Código}               % Para los captions (singular)
% \renewcommand{\lstlistlistingname}{\MakeUppercase{Índice de Códigos}} % Para el índice
% \lstlistoflistings


\newpage
\thispagestyle{empty}
\vspace*{0.35\textheight}
\begin{center}
	{\Huge\textbf{CAPÍTULO I}} \\[0.5cm]
	{\Huge\textbf{MARCO INTRODUCTORIO}}
\end{center}
\newpage
\thispagestyle{fancy}

\newpage
\pagenumbering{arabic}
\section*{}
\begin{center}
	\Large \textbf{INTRODUCCIÓN}
\end{center}

En los últimos años, la adopción de herramientas de Inteligencia Artificial (IA) en el  ámbito del desarrollo de software ha experimentado un crecimiento acelerado. La aparición de modelos generativos, asistentes de código y plataformas de automatización ha  transformado las dinámicas de trabajo de los desarrolladores, influyendo en productividad, competencias técnicas y procesos de toma de decisiones dentro de las organizaciones  tecnológicas.

El dataset \textit{Stack Overflow Annual Developer Survey 2025} constituye una de las  fuentes más completas y actuales sobre el perfil profesional, tecnológico y educativo de  desarrolladores a nivel global. Este conjunto de datos permite explorar de manera cuantitativa  los factores asociados a la adopción de herramientas de IA y genera un marco idóneo para  construir modelos predictivos basados en aprendizaje automático.

El propósito general de este estudio es desarrollar un modelo supervisado capaz de  predecir si un encuestado utiliza herramientas de IA, empleando variables profesionales, educativas y tecnológicas. Para ello, se realiza un proceso analítico que incluye limpieza de  datos, análisis exploratorio, selección de técnicas, comparación de modelos y validación  cuantitativa utilizando métricas de desempeño como F1-score y curvas ROC.

Metodológicamente, se aplican dos enfoques de clasificación ampliamente utilizados:  regresión logística y Random Forest. Ambos modelos se evalúan mediante validación cruzada  estratificada (k=5), considerando la naturaleza binaria de la variable objetivo y el leve  desbalance presente en el conjunto de datos. Asimismo, se analizan las ventajas, limitaciones  y estabilidad de cada modelo en función de su desempeño y su interpretabilidad.

Finalmente, este informe se estructura de la siguiente manera: en el Capítulo 2 se presenta  el marco teórico relacionado con el modelamiento supervisado y las métricas utilizadas; en el  Capítulo 3 se describe el conjunto de datos y se desarrollan los análisis exploratorios; en el  Capítulo 4 se exponen los resultados del modelamiento y la comparación entre técnicas; y en  el Capítulo 5 se presentan las conclusiones, implicaciones prácticas y recomendaciones derivadas del estudio.


\chapter{MARCO INTRODUCTORIO}

\section{Planteamiento del Problema y Pregunta de Investigación}
\subsection{Contextualización del Problema}
El uso de herramientas basadas en Inteligencia Artificial (IA) se ha expandido de manera acelerada en el ámbito del desarrollo de software. Sin embargo, los niveles de adopción varían sustancialmente entre diferentes perfiles profesionales, regiones, industrias y niveles educativos. Esto plantea la necesidad de identificar los factores que explican esta variabilidad y permiten anticipar patrones de adopción tecnológica.

El dataset público \textit{Stack Overflow Annual Developer Survey 2025} ofrece información estructurada sobre miles de desarrolladores a nivel global, permitiendo abordar este problema desde un enfoque cuantitativo y predictivo.

\subsection{Problema de Investigación}
A pesar de la disponibilidad de datos, no se conoce con claridad qué características del perfil de un desarrollador influyen más en la probabilidad de que utilice o no herramientas de IA. Tampoco se ha establecido un modelo sistemático que permita predecir este comportamiento y evaluar su desempeño mediante métricas apropiadas.

\subsection{Pregunta de Investigación Principal}

La pregunta central que guía el presente estudio es la siguiente:

\begin{quote}
\textbf{¿Podemos predecir la adopción de IA en desarrolladores utilizando su perfil profesional y tecnológico en 2025?}
\end{quote}

Esta formulación permite estructurar el análisis en torno a una variable objetivo claramente definida (\textit{AI\_Usage}) y a un conjunto de variables predictoras que combinan atributos categóricos y numéricos relevantes para el comportamiento tecnológico.

\subsection{Tipo de Pregunta Analítica}

De acuerdo con la naturaleza de la variable objetivo (\textit{usa IA} = Sí/No), el estudio se enmarca en una \textbf{pregunta analítica de tipo predictivo}. Esto implica que el propósito no es explicar causalmente por qué se adopta la IA, sino construir un modelo capaz de estimar la probabilidad de adopción utilizando patrones aprendidos a partir de los datos disponibles.

El problema se aborda mediante técnicas de \textbf{clasificación supervisada}, ya que el dataset incluye etiquetas reales que permiten entrenar modelos predictivos bajo un esquema de aprendizaje supervisado.

La pregunta planteada corresponde a un análisis de tipo:

\begin{itemize}
    \item \textbf{Predictivo}: se busca estimar un resultado futuro o no observado.
    \item \textbf{Supervisado}: la variable objetivo (\texttt{AI\_Usage}) está disponible y es binaria.
    \item \textbf{De clasificación}: el análisis requiere asignar una categoría (usa IA / no usa IA).
\end{itemize}

\subsection{Justificación del Enfoque de Modelamiento}
La elección de un enfoque supervisado de clasificación se justifica por las siguientes razones:

\begin{enumerate}
    \item El dataset contiene una \textbf{variable objetivo claramente definida} relacionada con el uso de herramientas de IA.
    \item La naturaleza de la variable objetivo (binaria) permite la aplicación de modelos como regresión logística y Random Forest.
    \item Existe interés práctico en \textbf{predecir la probabilidad de adopción} como apoyo a decisiones de formación, estrategias de talento y políticas internas en organizaciones tecnológicas.
    \item Las características de los desarrolladores (rol, país, industria, experiencia, educación) pueden representar factores relevantes y medibles para la predicción.
\end{enumerate}

\subsection{Objetivo}

Predecir la probabilidad de que un encuestado utilice herramientas de inteligencia artificial (\textit{AI Usage}: Sí/No) en función de sus características técnicas y demográficas.

\section{Hipótesis}

Un modelo predictivo permite identificar la tendencia de uso de IA en el desarrollo de software

\section{Criterio de Éxito}

El proyecto se considera exitoso si el modelo predictivo cumple con el siguiente criterio:

\begin{quote}
\textbf{F1-score mayor o igual a 0.80 en el conjunto de prueba, utilizando validación cruzada
con $k = 5$ para asegurar la estabilidad del desempeño.}
\end{quote}

\newpage
\thispagestyle{empty}
\vspace*{0.35\textheight}
\begin{center}
	{\Huge\textbf{CAPÍTULO III}} \\[0.5cm]
	{\Huge\textbf{DESCRIPCIÓN DEL CONJUNTO DE DATOS}}
\end{center}

\newpage
\chapter{DESCRIPCIÓN DEL CONJUNTO DE DATOS}

\section{Origen y Contexto del Dataset}

El análisis se desarrolla utilizando el \textit{Stack Overflow Annual Developer Survey 2025}, publicado en la plataforma Kaggle \parencite{kaggle2025}. El dataset original contiene \textbf{49\,123 filas} y \textbf{170 columnas}, incluyendo preguntas de respuesta múltiple, variables categóricas y campos abiertos.

\begin{center}
\url{https://www.kaggle.com/datasets/edoardogalli/stack-overflow-annual-developer-survey-2025}
\end{center}

Este conjunto de datos reúne respuestas de desarrolladores de todo el mundo y contiene información relevante sobre su formación, experiencia, tecnologías utilizadas, modalidad de trabajo y el uso actual de herramientas de inteligencia artificial. Su amplitud y enfoque en perfiles tecnológicos lo convierten en una fuente adecuada para estudiar factores asociados a la adopción de IA.

\section{Variables Incluidas}

Para este estudio se seleccionó un subconjunto de variables que presentan relevancia predictiva y consistencia conceptual respecto a la pregunta de investigación. La Tabla \ref{tab:variables_dataset} resume las variables empleadas:

\begin{table}[H]
\centering
\caption{Variables seleccionadas del dataset Stack Overflow 2025}
\label{tab:variables_dataset}
\begin{tabular}{p{3.5cm} p{10cm}}
\toprule
\textbf{Variable} & \textbf{Descripción} \\
\midrule
\texttt{AI\_Usage} & Variable objetivo (1 = usa IA, 0 = no usa IA). \\
\texttt{DevType} & Rol profesional del desarrollador. \\
\texttt{WorkExp} & Años de experiencia laboral. \\
\texttt{NumLanguages} & Recuento numérico derivado de la columna original LanguageHaveWorkedWith. \\
\texttt{Country} & País de residencia. \\
\texttt{RemoteWork} & Modalidad de trabajo (remoto, híbrido o presencial). \\
\texttt{Industry} & Industria o sector en el que trabaja el encuestado. \\
\texttt{OrgSize} & Tamaño de la organización. \\
\texttt{EdLevel} & Nivel de formación académica alcanzado. \\
\bottomrule
\end{tabular}
\end{table}

Estas variables combinan atributos categóricos y numéricos, lo que permite aplicar modelos de clasificación supervisada mediante estrategias de codificación y escalamiento.

\section{Variables Predictoras Utilizadas}

Las variables seleccionadas para el modelamiento se organizan por tipo conceptual, tal como se muestra en la Tabla \ref{tab:predictoras_usadas}. Esta clasificación facilita la interpretación del modelo y asegura coherencia con la literatura sobre adopción tecnológica.

\begin{table}[H]
\centering
\caption{Predictoras utilizadas y justificación conceptual}
\label{tab:predictoras_usadas}
\begin{tabular}{p{3.5cm} p{4cm} p{7cm}}
\toprule
\textbf{Tipo de variable} & \textbf{Columna} & \textbf{Justificación} \\
\midrule
Profesional     & \texttt{DevType}       & El rol del desarrollador influye en la adopción de IA. \\
Experiencia     & \texttt{WorkExp}       & Años de experiencia laboral. \\
Tecnológica     & \texttt{LanguageHaveWorkedWith} & Lenguajes utilizados (Python, R, etc.). \\
Geográfica      & \texttt{Country}       & Contexto regional y acceso a IA. \\
Laboral         & \texttt{RemoteWork}    & Modalidad de trabajo. \\
Sectorial       & \texttt{Industry}      & Tipo de industria o empresa. \\
Organizacional  & \texttt{OrgSize}       & Tamaño de la organización. \\
Educativa       & \texttt{EdLevel}       & Nivel de formación académica. \\
\bottomrule
\end{tabular}
\end{table}

\section{Procesamiento y Limpieza de Datos}

El dataset original descargado desde Kaggle contiene \textbf{49\,123 filas} y \textbf{170 columnas}, incluyendo múltiples variables textuales de respuesta múltiple, así como valores faltantes en distintas secciones del formulario. El primer paso del preprocesamiento consistió en evaluar la estructura global del dataset y la distribución de valores nulos.


\begin{itemize}
    \item Eliminación de filas con valores faltantes en la variable objetivo \texttt{AI\_Usage}.
		\item Imputación de valores faltantes en \texttt{WorkExp} mediante la mediana.
    \item Imputación de valores faltantes en variables categóricas usando la categoría genérica \textit{``No especificado''}.
    \item Normalización del formato de variables categóricas.
    \item Conversión de la columna \texttt{LanguageHaveWorkedWith} en un recuento numérico denominado \texttt{NumLanguages}.
    \item Estandarización de variables numéricas mediante \textit{StandardScaler}.
    \item Codificación \textit{One-Hot Encoding} para todas las variables categóricas.
\end{itemize}

Estas transformaciones son consistentes con buenas prácticas de preprocesamiento para modelos de clasificación.

\subsection{Inspección de Valores Faltantes}

La Figura \ref{fig:missing_matrix} presenta la matriz de valores faltantes generada a partir de
las variables seleccionadas para el estudio. Esta visualización permite identificar patrones
sistemáticos de ausencia de datos y guiar las decisiones de imputación o eliminación.

\begin{figure}[H]
\centering
\includegraphics[width=\textwidth]{images/02_valores_faltantes.png}
\caption{Matriz de valores faltantes en las variables predictoras seleccionadas.}
\label{fig:missing_matrix}
\end{figure}

Tras esta inspección se identificó que algunas columnas contenían valores faltantes esporádicos, especialmente en \texttt{DevType}, \texttt{Industry} y \texttt{RemoteWork}. Sin embargo, la variable objetivo derivada (\texttt{AI\_Usage}) debía estar completamente observada para mantener la coherencia del enfoque supervisado. Por ello se decidió:

\begin{itemize}
    \item Eliminar filas con valores faltantes en la variable objetivo.
    \item Conservar filas con valores faltantes en columnas predictoras, dado que su proporción
    era baja y podían manejarse mediante codificación categórica.
\end{itemize}

Después del filtrado inicial, el dataset quedó reducido a \textbf{33\,686 filas} y \textbf{9 columnas} (selección final de predictoras + variable objetivo).

\subsection{Construcción de la Variable Objetivo \texttt{AI\_Usage}}

La variable objetivo \texttt{AI\_Usage} fue construida a partir de la pregunta original del formulario: \textit{``Do you currently use AI tools in your development process?''}. Esta pregunta incluye cinco opciones que representan distintos niveles de adopción de herramientas de inteligencia artificial:

\begin{itemize}
    \item Yes, I use AI tools daily.
    \item Yes, I use AI tools weekly.
    \item Yes, I use AI tools monthly or infrequently.
    \item No, but I plan to soon.
    \item No, and I don't plan to.
\end{itemize}

Para los fines del presente estudio, estas categorías fueron agrupadas con el objetivo de crear una variable binaria adecuada para técnicas de clasificación supervisada. El criterio de binarización fue el siguiente:

\begin{itemize}
    \item \textbf{AI\_Usage = 1}: participantes que declararon utilizar herramientas de IA de manera diaria, semanal o mensual.
    \item \textbf{AI\_Usage = 0}: participantes que no utilizan IA o no planean utilizarla.
\end{itemize}

Esta transformación permite convertir una pregunta de escala múltiple en una variable objetivo consistente y directamente alineada con la pregunta de investigación: predecir si un desarrollador utiliza o no herramientas de inteligencia artificial.

Esta variable fue verificada para asegurar que no quedaran valores faltantes tras la binarización y que las proporciones de clases fueran adecuadas para el modelamiento.



\section{Justificación de Exclusiones y Transformaciones}

Se excluyeron variables redundantes o con granularidad excesiva (como listas textuales extremamente largas) para asegurar un procesamiento eficiente y evitar problemas de dimensionalidad. La variable objetivo \texttt{AI\_Usage} se mantuvo como binaria porque responde directamente a la pregunta planteada y permite implementar un enfoque predictivo sin ambigüedad conceptual.

Asimismo, la transformación de lenguajes de programación a un recuento (\texttt{NumLanguages}) se justifica por su contribución a la interpretabilidad del modelo y la reducción de ruido.

\section{Visualización Inicial del Dataset}

Se realizaron gráficos exploratorios univariados y bivariados para identificar patrones iniciales y relaciones potenciales entre las variables predictoras y la variable objetivo. Estos gráficos incluyen distribuciones de experiencia laboral, relación entre sector e \texttt{AI\_Usage}, modalidad de trabajo y correlación mixta. Las visualizaciones completas se presentan en el capítulo correspondiente al análisis exploratorio.
\newpage
\thispagestyle{empty}
\vspace*{0.35\textheight}
\begin{center}
	{\Huge\textbf{CAPÍTULO III}} \\[0.5cm]
	{\Huge\textbf{MARCO METODOLÓGICO}}
\end{center}

\newpage
\chapter{MARCO METODOLÓGICO}

\section{Tipo de Investigación}

La presente investigación es de tipo \textbf{aplicada y cuantitativa}. Se considera \textbf{aplicada} porque busca resolver un problema específico mediante la implementación de un modelo computacional. Es \textbf{cuantitativa} porque se basa en la medición numérica de características lingüísticas de letras de canciones y en el cálculo de métricas estadísticas para validar el desempeño del modelo.

Asimismo, se enmarca en un diseño \textbf{no experimental, transversal y explicativo}. Es \textbf{no experimental} al no manipular variables sino analizar datos ya existentes; \textbf{transversal} (corte único temporal) al procesar el corpus completo de 1990--2022 en un solo análisis —a diferencia de un diseño \textbf{longitudinal}, que implicaría múltiples mediciones a lo largo del tiempo para observar cambios en las mismas instancias—; y \textbf{explicativo} porque busca identificar relaciones causales entre variables lingüísticas (p. ej., uso de n-gramas o términos afectivos) y el éxito musical medido por premiaciones.

\section{Diseño Metodológico}

El estudio adopta el ciclo \textbf{CRISP-DM} (Cross-Industry Standard Process for Data Mining) por su carácter iterativo y flexible en proyectos de minería de datos. Las fases implementadas fueron:

\begin{itemize}
  \item \textbf{Comprensión del problema y de los datos:} Definición de variables, revisión documental y recolección de letras y metadatos de premiación.
  \item \textbf{Preparación de datos:} Limpieza textual (remoción de ruido, stopwords), tokenización, lematización y vectorización con TF-IDF para cuantificar características léxicas y semánticas.
  \item \textbf{Modelado:} Entrenamiento de algoritmos supervisados (Naive Bayes, Regresión Logística, SVM, Random Forest, XGBoost) con validación cruzada estratificada, lo que garantiza la representatividad de cada clase en los pliegues.
  \item \textbf{Evaluación:} Medición de desempeño a través de métricas estadísticas (accuracy, precision, recall, F1-score) y análisis de matrices de confusión para determinar errores de clasificación.
  \item \textbf{Despliegue parcial:} Generación de visualizaciones (curvas ROC, nubes de palabras) y discusión de resultados para extraer conclusiones prácticas.
\end{itemize}

Se eligió CRISP-DM por permitir iterar entre fases de preparación y modelado tras cada ronda de evaluación, optimizando así la calidad del modelo y reduciendo el sesgo.

\section{Técnicas e Instrumentos de Recolección de Datos}

Los datos fueron recolectados a partir de fuentes documentales digitales. Las letras de canciones fueron obtenidas mediante scraping desde el sitio \textit{Genius.com}, mientras que la información sobre premiación fue extraída del repositorio oficial de la \textbf{Recording Industry Association of America (RIAA)}. La selección se centró en canciones en español del periodo 1990--2022.

Se diseñó un dataset balanceado artificialmente a través de \textbf{undersampling} para evitar el sesgo por clase (premiada vs no premiada).

\section{Población y Muestra}

\subsection{Población}

La población de estudio está compuesta por letras de canciones en idioma español publicadas entre los años 1990 y 2022. Esta población incluye tanto obras musicales que han recibido algún tipo de reconocimiento oficial, como aquellas que no han sido premiadas, pero que cuentan con presencia pública en plataformas digitales.

Se considera que las letras reflejan una dimensión lingüística clave del producto musical y pueden contener patrones asociados al éxito o impacto cultural. La población está implícitamente delimitada por la accesibilidad pública de las letras en bases como \textit{Genius.com}, y por la disponibilidad de información sobre premiación otorgada por la \textit{Recording Industry Association of America (RIAA)}.

\subsection{Muestra}

La muestra fue compuesta por un total de aproximadamente \textbf{5.678 canciones en español}, organizadas en dos clases:

\begin{itemize}
    \item \textbf{Premiadas:} canciones en español que figuran en los registros públicos de la RIAA y han sido reconocidas con discos de Oro, Platino o Multiplatino.
    \item \textbf{No premiadas:} canciones en español sin registro de premiación en la RIAA, seleccionadas de forma equilibrada desde fuentes abiertas.
\end{itemize}

\subsection{Criterios de selección}

Para asegurar la relevancia y comparabilidad entre canciones, se aplicaron los siguientes criterios:

\begin{itemize}
    \item Canciones en idioma español (total o mayoritariamente).
    \item Publicadas entre 1990 y 2022.
    \item Letras disponibles de forma pública y en formato textual procesable.
    \item Exclusión de remixes, versiones instrumentales o duplicadas.
\end{itemize}

\subsection{Estrategia de muestreo}

Se empleó un muestreo \textbf{no probabilístico por conveniencia}, dado que se seleccionaron canciones a partir de su disponibilidad en plataformas específicas. No obstante, se implementaron acciones para asegurar \textbf{balance y diversidad}:

\begin{itemize}
    \item Igual número de canciones premiadas y no premiadas para evitar sesgo por clase.
    \item Inclusión de distintos géneros musicales (pop, reguetón, rock, balada, etc.).
    \item Distribución temporal amplia (años 90, 2000s, 2010s y 2020s).
\end{itemize}

Esta estrategia buscó garantizar la representatividad del fenómeno musical en español durante el periodo analizado, sin comprometer la calidad del análisis computacional posterior.
\section{Técnicas de Análisis de Datos}

Se aplicaron las siguientes técnicas:

\begin{itemize}
    \item \textbf{Procesamiento de Lenguaje Natural (PLN):} limpieza, lematización y vectorización con TF-IDF.
    \item \textbf{Aprendizaje supervisado:} uso de algoritmos de clasificación binaria.
    \item \textbf{Evaluación estadística:} cálculo de métricas de desempeño del modelo (accuracy, precision, recall, F1-score) y matrices de confusión.
    \item \textbf{Visualización:} nubes de palabras, gráficos de boxplot, curvas ROC.
\end{itemize}

\section{Identificación de Variables}

En la presente investigación se han identificado las siguientes variables clave, según su función dentro del diseño metodológico:

\newcolumntype{Y}{>{\raggedright\arraybackslash}X}
\begin{table}[H]
\centering
\caption{Identificación de Variables}
\begin{tabularx}{\textwidth}{|Y|Y|Y|Y|}
\hline
\textbf{Nombre de la variable} & \textbf{Tipo} & \textbf{Descripción} & \textbf{Escala de medición} \\
\hline
\textbf{Contenido lingüístico de las letras} & Independiente & Características léxicas y semánticas de las letras de canciones en español. & Ordinal (frecuencia, puntuaciones TF-IDF) \\
\hline
\textbf{Éxito musical (premiación)} & Dependiente & Condición de reconocimiento oficial otorgado por la RIAA a las canciones. & Nominal dicotómica (Premiada = 1, No premiada = 0) \\
\hline
\textbf{Desempeño del modelo clasificatorio} & Auxiliar & Capacidad del modelo para predecir correctamente si una canción es premiada o no. & De razón (porcentaje de F1-score, accuracy) \\
\hline
\end{tabularx}
\label{tab:identificacion_variables}
\end{table}

\section{Operacionalización de Variables}

La siguiente tabla resume las variables principales consideradas en la presente investigación, su tipo, dimensiones e indicadores, así como los instrumentos utilizados para su medición y análisis.

\newcolumntype{Y}{>{\raggedright\arraybackslash}X}

\begin{table}[H]
\centering
\caption{Tabla de Operacionalización de Variables}
\begin{tabularx}{\textwidth}{|Y|Y|Y|Y|Y|}
\hline
\textbf{Variable} & \textbf{Definición conceptual} & \textbf{Dimensiones} & \textbf{Indicadores} & \textbf{Instrumento} \\
\hline
\textbf{Contenido lingüístico de las letras (Independiente)} & Conjunto de elementos léxicos y semánticos presentes en letras de canciones en español. & Nivel léxico & Frecuencia de palabras, n-gramas, diversidad léxica & Script de procesamiento (Python + spaCy) \\
\cline{3-5}
& & Nivel semántico & Palabras afectivas, temas líricos, lematización & Modelo semántico, Wordcloud \\
\hline
\textbf{Éxito musical (premiación) (Dependiente)} & Reconocimiento institucional otorgado por la RIAA. & Clase de premiación & Premiación (1/0), tipo de premio & Base de datos RIAA \\
\hline
\textbf{Desempeño del modelo (Auxiliar)} & Nivel de acierto del modelo de clasificación supervisada. & Precisión predictiva & F1-score, accuracy, matriz de confusión & Scikit-learn, validación cruzada \\
\hline
\end{tabularx}
\label{tab:operacionalizacion}
\end{table}


\section{Tabla de Consistencia}

A continuación se presenta la matriz de consistencia metodológica que vincula los componentes fundamentales del estudio:

\begin{figure}[ht]
  \centering
  \includegraphics[width=\textwidth]{02_matriz_consistencia.png}
  \caption{Matriz de consistencia}
  \label{fig:matriz_consistencia}
\end{figure}

\section{Etapas del Estudio}

\begin{enumerate}
    \item Revisión documental del problema y casos similares.
    \item Recolección y depuración de datos textuales.
    \item Preprocesamiento con herramientas de PLN.
    \item Entrenamiento y validación cruzada de modelos supervisados.
    \item Análisis comparativo de métricas.
    \item Interpretación de resultados y discusión.
\end{enumerate}

\section{Cronograma}

En la tabla~\ref{tab:cronograma} se describen las fases del proyecto según CRISP-DM, junto con las actividades clave y sus fechas de inicio y fin. El cronograma abarca un año, desde julio de 2025 hasta junio de 2026.

\begin{table}[htbp]
  \centering
  \caption{Cronograma de actividades y fases CRISP-DM}
  \label{tab:cronograma}
  \begin{tabular}{@{}p{7cm}cc@{}}
    \toprule
    \textbf{Actividad}                           & \textbf{Inicio}      & \textbf{Fin}         \\
    \midrule
    \multicolumn{3}{@{}l}{\textbf{Fases CRISP-DM}} \\[0.2em]
    Comprensión del negocio                      & Julio 2025           & Agosto 2025          \\
    Comprensión de datos                         & Septiembre 2025      & Octubre 2025         \\
    Preparación de datos                         & Noviembre 2025       & Diciembre 2025       \\
    Modelado                                     & Enero 2026           & Febrero 2026         \\
    Evaluación                                   & Marzo 2026           & Abril 2026           \\
    Despliegue                                   & Mayo 2026            & Junio 2026           \\
    \addlinespace
    \multicolumn{3}{@{}l}{\textbf{Otras actividades}} \\[0.2em]
    Revisión bibliográfica                       & Julio 2025           & Agosto 2025          \\
    Diseño de instrumentos                       & Septiembre 2025      & Octubre 2025         \\
    Trabajo de campo / recolección de datos      & Noviembre 2025       & Diciembre 2025       \\
    Análisis de datos                            & Enero 2026           & Febrero 2026         \\
    Redacción de resultados                      & Marzo 2026           & Abril 2026           \\
    Revisión y corrección final                  & Mayo 2026            & Junio 2026           \\
    \bottomrule
  \end{tabular}
\end{table}


\section{Presupuesto}

\subsection{Recursos informáticos}

Para el desarrollo del proyecto se requerirá un entorno local basado en Ubuntu 24.04, sin necesidad de GPU, aprovechando hardware de gama media. El equipamiento previsto es el siguiente:

\begin{table}[H]
  \centering
  \caption{Características de hardware y software}
  \label{tab:recursos_informaticos}
  \begin{tabular}{|p{8cm}|p{8cm}|}
    \hline
    \textbf{Hardware}                         & \textbf{Software}               \\
    \hline
    Procesador: Intel Core i7-4700MQ          & Sistema operativo: Ubuntu 24.04 \\
    Memoria RAM: 16 GB                        & Python: v3.12.3                 \\
    Almacenamiento: 1 TB SSD (800 GB libre)   & Entorno virtual: Sí             \\
    Tarjeta madre: HP Envy 1968               & Jupyter Notebook: v8.1.0        \\
    \hline
  \end{tabular}
\end{table}

\subsection{Presupuesto estimado}

Dado que todo el software empleado será de código abierto y no requerirá licencias, el presupuesto se centrará en honorarios de expertos para la validación de resultados y en el equipo de cómputo en caso de no tenerse:

\begin{table}[H]
  \centering
  \caption{Presupuesto estimado}
  \label{tab:presupuesto_estimated}
  \begin{tabular}{|p{12cm}|r|}
    \hline
    \textbf{Concepto}                                                    & \textbf{Costo (Bs)} \\
    \hline
    Honorarios para expertos (encuesta de validación)                    & 1\,000              \\
    \hline
    Equipo de cómputo (Intel i7, 16 GB RAM, SSD o laptop equivalente)     & 8\,000              \\
    \hline
    \textbf{Total}                                                       & \textbf{9\,000}     \\
    \hline
  \end{tabular}
\end{table}

\newpage
\thispagestyle{empty}
\vspace*{0.35\textheight}
\begin{center}
	{\Huge\textbf{CAPÍTULO IV}} \\[0.5cm]
	{\Huge\textbf{MARCO APLICATIVO}}
\end{center}

\newpage
\chapter{MARCO APLICATIVO}

\section{Descripción del dataset}
    \subsection{Características de las letras de canciones}
    \subsection{Datos RIAA y premiación}
\section{Preprocesamiento de datos}
    \subsection{Limpieza textual}
    \subsection{Tokenización, lematización y stopwords}
    \subsection{Vectorización con TF-IDF}
\section{Modelado supervisado}
    \subsection{Modelos aplicados}
    \subsubsection{Naive Bayes}
    \subsubsection{Regresión logística}
    \subsubsection{SVM, Random Forest, XGBoost}
    \subsection{Métricas de evaluación}
\section{Análisis de resultados}
    \subsection{Comparación entre modelos}
    \subsection{Análisis de errores}
    \subsection{Discusión crítica}

\newpage
\thispagestyle{empty}
\vspace*{0.35\textheight}
\begin{center}
	{\Huge\textbf{CAPÍTULO VI}} \\[0.5cm]
	{\Huge\textbf{CONCLUSIONES Y}}\\[0.5cm]
    {\Huge\textbf{RECOMENDACIONES}}
\end{center}

\newpage
\chapter{CONCLUSIONES Y RECOMENDACIONES}

\section{Conclusiones}

El presente proyecto logró construir y evaluar un modelo supervisado capaz de predecir la adopción de herramientas de inteligencia artificial entre desarrolladores, utilizando datos de la encuesta global \textit{Stack Overflow Developer Survey 2025}. A través de un proceso metodológico completo —comprendiendo exploración, selección de variables, comparación de modelos y optimización— se obtuvieron resultados consistentes y alineados con el criterio de éxito definido en el Sprint.

En primer lugar, el análisis exploratorio mostró que la adopción de IA no depende de un único factor dominante, sino de combinaciones de características profesionales, tecnológicas y contextuales. Variables como \textbf{DevType}, \textbf{Country}, \textbf{Industry} y \textbf{OrgSize} presentaron correlaciones moderadas con el uso de IA, mientras que \textbf{WorkExp} y \textbf{EdLevel} mostraron contribuciones más leves. Esta estructura multifactorial justificó el uso de un modelo supervisado capaz de manejar tanto variables categóricas como numéricas.

La comparación de técnicas predictivas permitió evaluar el desempeño de \textbf{Regresión Logística} y \textbf{Random Forest} en versiones balanceadas y no balanceadas. Ambos modelos superaron el criterio de éxito (F1 $\geq 0.80$), pero la \textbf{Regresión Logística sin balance} destacó por su rendimiento competitivo (F1 $\approx0.8817$), su baja variabilidad entre folds (0.0015) y su interpretabilidad, factores clave para análisis explicativos y toma de decisiones.

El proceso de optimización confirmó que el modelo baseline ya capturaba la mayor parte de la estructura del problema. La búsqueda exhaustiva con \textit{GridSearchCV} y la búsqueda aleatoria con \textit{RandomizedSearchCV} convergieron en hiperparámetros similares, alcanzando un F1 promedio de $\approx0.883$. Si bien la mejora respecto al modelo base fue modesta, el ajuste refinó significativamente el \textbf{recall} de la clase positiva, reduciendo falsos negativos de forma sustancial (105→17), lo que resulta valioso para escenarios donde omitir usuarios de IA representa un mayor riesgo que clasificarlos erróneamente como tales.

En términos generales, el modelo final ofrece un equilibrio adecuado entre rendimiento, estabilidad, eficiencia computacional y capacidad de interpretación. No obstante, persisten limitaciones propias del dataset, como la naturaleza autodeclarada de la encuesta y la variabilidad cultural asociada a los países participantes. Asimismo, el AUC moderado ($\approx0.67$) indica que aún existe superposición entre clases, lo que podría explorarse en trabajos futuros mediante modelos más complejos o nuevas fuentes de información.

En síntesis, se concluye que la \textbf{Regresión Logística optimizada} constituye una solución sólida y confiable para predecir la adopción de IA con las variables disponibles. El modelo proporciona insights relevantes para comprender qué perfiles profesionales, regiones e industrias presentan mayor propensión a integrar herramientas de IA, lo cual puede orientar estrategias de formación, políticas de talento y decisiones organizacionales en un contexto tecnológico en rápida evolución.

\section{Limitaciones}

A pesar de los resultados obtenidos, el presente estudio presenta varias limitaciones que deben considerarse al interpretar las conclusiones:

\begin{itemize}
    \item \textbf{Naturaleza autodeclarada del dataset.} La encuesta Stack Overflow 2025 se basa en respuestas voluntarias, lo que introduce sesgos de percepción, autoselección y representatividad desigual entre países.

    \item \textbf{Variables limitadas respecto al dataset original.} El análisis utilizó un subconjunto reducido de variables, lo que puede omitir factores relevantes asociados al uso de IA (por ejemplo, exposición a herramientas específicas, motivaciones personales o cultura organizacional).

    \item \textbf{Correlaciones bajas entre predictores y la variable objetivo.} La mayoría de las asociaciones cuantitativas fueron débiles, lo que reduce la capacidad teórica de separación entre clases y explica el AUC moderado ($\approx 0.67$).

    \item \textbf{Posible ruido en variables categóricas de alta cardinalidad.} Categorías como \textit{DevType} o \textit{Country} pueden presentar ruido semántico o dispersión significativa, afectando la estabilidad del modelo.

    \item \textbf{Ausencia de análisis temporal.} El estudio es transversal y no permite evaluar la evolución del uso de IA a lo largo del tiempo.
\end{itemize}

Estas limitaciones no invalidan los resultados, pero sí sugieren cautela al generalizar los hallazgos y señalan oportunidades para mejorar el modelo en trabajos futuros.

\section{Recomendaciones y Trabajo Futuro}

Con base en los resultados del modelo final y en las limitaciones identificadas, se proponen las siguientes recomendaciones para estudios posteriores y aplicaciones prácticas:

\subsection*{Recomendaciones para organizaciones o tomadores de decisiones}

\begin{itemize}
    \item \textbf{Enfocar programas de formación en perfiles técnicos clave.} Roles como científicos de datos, ingenieros de software y desarrolladores especializados muestran mayor propensión a adoptar IA; fortalecer su capacitación puede acelerar procesos de innovación.

    \item \textbf{Priorizar estrategias de adopción según industria y región.} Sectores como software, telecomunicaciones y fintech presentan mayor afinidad con herramientas de IA, mientras que otros requieren políticas más específicas.

    \item \textbf{Utilizar modelos explicables para toma de decisiones.} La Regresión Logística permite identificar qué factores influyen en la adopción, facilitando planes de formación y políticas organizacionales basadas en evidencia.
\end{itemize}

\subsection*{Trabajo futuro}

\begin{itemize}
    \item \textbf{Incorporar nuevas variables del dataset completo}, como tecnologías específicas utilizadas, tipo de proyecto, motivaciones personales y nivel de exposición a IA.

    \item \textbf{Evaluar modelos más complejos}, tales como XGBoost, LightGBM o redes neuronales, para explorar si pueden mejorar el AUC y la capacidad de discriminación.

    \item \textbf{Experimentar con métodos de \textit{feature engineering}}: extracción de tópicos en descripciones, codificación semántica de roles o embeddings de países e industrias.

    \item \textbf{Realizar análisis temporal} con ediciones anteriores y futuras de la encuesta para identificar tendencias en adopción de IA a lo largo del tiempo.

    \item \textbf{Explorar técnicas de calibración de probabilidades}, como Platt scaling o isotonic regression, para mejorar la interpretación probabilística del modelo.

    \item \textbf{Implementar el modelo en un entorno productivo} (API o dashboard) para monitorear la adopción de IA en diversas poblaciones o empresas.
\end{itemize}

Estas recomendaciones buscan fortalecer la capacidad predictiva del modelo y ampliar su aplicación práctica en contextos reales de toma de decisiones.

\newpage
\printbibliography[
	heading=bibintoc,         % Agrega al índice
	title={BIBLIOGRAFÍA}      % Cambia el título
]
% \include{6_anexos}
\end{document}