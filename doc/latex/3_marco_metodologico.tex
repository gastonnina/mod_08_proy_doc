\newpage
\thispagestyle{empty}
\vspace*{0.35\textheight}
\begin{center}
	{\Huge\textbf{CAPÍTULO III}} \\[0.5cm]
	{\Huge\textbf{ANÁLISIS EXPLORATORIO DE DATOS}}
\end{center}

\newpage
\chapter{ANÁLISIS EXPLORATORIO DE DATOS (EDA)}

El análisis exploratorio de datos (EDA) tiene como objetivo identificar patrones, tendencias y relaciones entre las variables predictoras y la variable objetivo \texttt{AI\_Usage}. 

Este capítulo se organiza en tres partes:
(1) análisis bivariado,
(2) matriz de correlación mixta, y
(3) descripción de la naturaleza de los campos del dataset limpio.


\section{Análisis Bivariado}

Las siguientes visualizaciones muestran cómo se comportan las variables predictoras en relación con la variable objetivo \texttt{AI\_Usage}. Todas las figuras provienen del dataset ya procesado y limpio.

\subsection{Uso de IA según experiencia laboral (\texttt{WorkExp})}

\begin{figure}[H]
\centering
\includegraphics[width=\textwidth]{images/03_eda_1_workexp.png}
\caption{Distribución de años de experiencia según uso de IA.}
\label{fig:eda_workexp}
\end{figure}

La mediana de experiencia laboral es similar entre usuarios y no usuarios de IA.  Sin embargo, se observan \textit{outliers} de alta experiencia dentro del grupo que sí utiliza IA, lo cual sugiere que ciertos perfiles muy senior adoptan herramientas de IA con mayor frecuencia.

\subsection{Uso de IA según número de lenguajes dominados (\texttt{NumLanguages})}

\begin{figure}[H]
\centering
\includegraphics[width=\textwidth]{images/03_eda_2_numlanguages.png}
\caption{Distribución del número de lenguajes según uso de IA.}
\label{fig:eda_numlanguages}
\end{figure}

Ambos grupos presentan distribuciones similares, aunque los usuarios de IA muestran  una ligera tendencia a manejar un mayor número de lenguajes, indicando mayor  diversidad tecnológica.

\subsection{Adopción de IA por país (\texttt{Country})}

\begin{figure}[H]
\centering
\includegraphics[width=\textwidth]{images/03_eda_3_country.png}
\caption{Mapa mundial de usuarios de IA.}
\label{fig:eda_country}
\end{figure}

La adopción de IA se concentra en polos tecnológicos globales como Estados Unidos, India, Reino Unido y Alemania. Este patrón sugiere que el contexto regional influye en la probabilidad de utilizar herramientas de IA.

\subsection{Uso de IA por industria (\texttt{Industry})}

\begin{figure}[H]
\centering
\includegraphics[width=\textwidth]{images/03_eda_4_industry.png}
\caption{Proporción de uso de IA por industria (Top 15).}
\label{fig:eda_industry}
\end{figure}

Las industrias tecnológicas (como software, fintech, telecomunicaciones e IT services) muestran los niveles más altos de adopción de IA. Sectores tradicionales como gobierno o educación muestran menor proporción relativa.

\subsection{Uso de IA por nivel educativo (\texttt{EdLevel})}

\begin{figure}[H]
\centering
\includegraphics[width=\textwidth]{images/03_eda_5_edlevel.png}
\caption{Proporción de uso de IA según nivel educativo.}
\label{fig:eda_edlevel}
\end{figure}

La adopción de IA aumenta entre quienes poseen formación universitaria, posgrado o grados profesionales. También se observa uso moderado entre perfiles técnicos y de
estudios incompletos.

\section{Matriz de Correlación Mixta}

\begin{figure}[H]
\centering
\includegraphics[width=\textwidth]{images/03_mixed_correlation_matrix.png}
\caption{Matriz de correlación mixta (numéricas y categóricas).}
\label{fig:correlacion_mixta}
\end{figure}

La matriz combina:
\begin{itemize}
    \item Pearson para variables numéricas.
    \item Cramér's V para variables categóricas.
    \item Correlación biserial para la variable objetivo binaria.
\end{itemize}

Los valores revelan que:

\begin{itemize}
    \item La correlación directa con \texttt{AI\_Usage} es baja en todas las variables, lo cual es habitual en datasets de comportamiento humano.
    \item Destacan como predictoras relevantes: \texttt{DevType} (0.19), \texttt{Country} (0.17), \texttt{Industry} (0.12) y \texttt{OrgSize} (0.09).
    \item Las asociaciones entre categóricas (\texttt{RemoteWork}, \texttt{Industry}, \texttt{OrgSize}) muestran patrones contextuales importantes.
\end{itemize}

Aunque las correlaciones no son fuertes, sí revelan señales relevantes para un modelo predictivo, especialmente para algoritmos no lineales como Random Forest.

\section{Naturaleza de los Campos del Dataset Limpio}

La Tabla \ref{tab:metadata_dataset} resume la naturaleza de cada campo utilizado en el análisis y posterior modelamiento.

\begin{table}[H]
\centering
\caption{Metadatos del dataset limpio}
\label{tab:metadata_dataset}
\begin{tabular}{p{3.5cm} p{4cm} p{7cm}}
\toprule
\textbf{Columna} & \textbf{Tipo (pandas)} & \textbf{Naturaleza} \\
\midrule
DevType & category & Categórica nominal (rol profesional). \\
WorkExp & float64 & Numérica continua (años de experiencia). \\
Country & category & Categórica nominal (país). \\
RemoteWork & category & Categórica ordinal (grado de trabajo remoto). \\
Industry & category & Categórica nominal (sector/industria). \\
OrgSize & category & Categórica ordinal (rango de tamaño de empresa). \\
EdLevel & category & Categórica ordinal (nivel educativo). \\
AI\_Usage & int64 & Binaria (variable objetivo: 0/1). \\
NumLanguages & int64 & Numérica discreta (conteo de lenguajes). \\
\bottomrule
\end{tabular}
\end{table}

\section{Síntesis del EDA}

\begin{itemize}
    \item El uso de IA muestra patrones claros asociados al tipo de rol, industria, país y nivel educativo.
    \item No existe una correlación lineal fuerte con ninguna variable individual, lo que sugiere que el fenómeno es multifactorial.
    \item El comportamiento de las variables respalda el uso de modelos de clasificación capaces de capturar relaciones no lineales.
\end{itemize}

En conjunto, el EDA proporciona la base analítica necesaria para proceder con la etapa de modelamiento presentada en el siguiente capítulo.