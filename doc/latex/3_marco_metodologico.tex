\newpage
\thispagestyle{empty}
\vspace*{0.35\textheight}
\begin{center}
	{\Huge\textbf{CAPÍTULO III}} \\[0.5cm]
	{\Huge\textbf{MARCO METODOLÓGICO}}
\end{center}

\newpage
\chapter{MARCO METODOLÓGICO}

\section{Tipo de Investigación}

La presente investigación es de tipo \textbf{aplicada y cuantitativa}. Se considera \textbf{aplicada} porque busca resolver un problema específico mediante la implementación de un modelo computacional. Es \textbf{cuantitativa} porque se basa en la medición numérica de características lingüísticas de letras de canciones y en el cálculo de métricas estadísticas para validar el desempeño del modelo.

Asimismo, se enmarca en un diseño \textbf{no experimental, transversal y explicativo}. Es \textbf{no experimental} al no manipular variables sino analizar datos ya existentes; \textbf{transversal} (corte único temporal) al procesar el corpus completo de 1990--2022 en un solo análisis —a diferencia de un diseño \textbf{longitudinal}, que implicaría múltiples mediciones a lo largo del tiempo para observar cambios en las mismas instancias—; y \textbf{explicativo} porque busca identificar relaciones causales entre variables lingüísticas (p. ej., uso de n-gramas o términos afectivos) y el éxito musical medido por premiaciones.

\section{Diseño Metodológico}

El estudio adopta el ciclo \textbf{CRISP-DM} (Cross-Industry Standard Process for Data Mining) por su carácter iterativo y flexible en proyectos de minería de datos. Las fases implementadas fueron:

\begin{itemize}
  \item \textbf{Comprensión del problema y de los datos:} Definición de variables, revisión documental y recolección de letras y metadatos de premiación.
  \item \textbf{Preparación de datos:} Limpieza textual (remoción de ruido, stopwords), tokenización, lematización y vectorización con TF-IDF para cuantificar características léxicas y semánticas.
  \item \textbf{Modelado:} Entrenamiento de algoritmos supervisados (Naive Bayes, Regresión Logística, SVM, Random Forest, XGBoost) con validación cruzada estratificada, lo que garantiza la representatividad de cada clase en los pliegues.
  \item \textbf{Evaluación:} Medición de desempeño a través de métricas estadísticas (accuracy, precision, recall, F1-score) y análisis de matrices de confusión para determinar errores de clasificación.
  \item \textbf{Despliegue parcial:} Generación de visualizaciones (curvas ROC, nubes de palabras) y discusión de resultados para extraer conclusiones prácticas.
\end{itemize}

Se eligió CRISP-DM por permitir iterar entre fases de preparación y modelado tras cada ronda de evaluación, optimizando así la calidad del modelo y reduciendo el sesgo.

\section{Técnicas e Instrumentos de Recolección de Datos}

Los datos fueron recolectados a partir de fuentes documentales digitales. Las letras de canciones fueron obtenidas mediante scraping desde el sitio \textit{Genius.com}, mientras que la información sobre premiación fue extraída del repositorio oficial de la \textbf{Recording Industry Association of America (RIAA)}. La selección se centró en canciones en español del periodo 1990--2022.

Se diseñó un dataset balanceado artificialmente a través de \textbf{undersampling} para evitar el sesgo por clase (premiada vs no premiada).

\section{Población y Muestra}

\subsection{Población}

La población de estudio está compuesta por letras de canciones en idioma español publicadas entre los años 1990 y 2022. Esta población incluye tanto obras musicales que han recibido algún tipo de reconocimiento oficial, como aquellas que no han sido premiadas, pero que cuentan con presencia pública en plataformas digitales.

Se considera que las letras reflejan una dimensión lingüística clave del producto musical y pueden contener patrones asociados al éxito o impacto cultural. La población está implícitamente delimitada por la accesibilidad pública de las letras en bases como \textit{Genius.com}, y por la disponibilidad de información sobre premiación otorgada por la \textit{Recording Industry Association of America (RIAA)}.

\subsection{Muestra}

La muestra fue compuesta por un total de aproximadamente \textbf{5.678 canciones en español}, organizadas en dos clases:

\begin{itemize}
    \item \textbf{Premiadas:} canciones en español que figuran en los registros públicos de la RIAA y han sido reconocidas con discos de Oro, Platino o Multiplatino.
    \item \textbf{No premiadas:} canciones en español sin registro de premiación en la RIAA, seleccionadas de forma equilibrada desde fuentes abiertas.
\end{itemize}

\subsection{Criterios de selección}

Para asegurar la relevancia y comparabilidad entre canciones, se aplicaron los siguientes criterios:

\begin{itemize}
    \item Canciones en idioma español (total o mayoritariamente).
    \item Publicadas entre 1990 y 2022.
    \item Letras disponibles de forma pública y en formato textual procesable.
    \item Exclusión de remixes, versiones instrumentales o duplicadas.
\end{itemize}

\subsection{Estrategia de muestreo}

Se empleó un muestreo \textbf{no probabilístico por conveniencia}, dado que se seleccionaron canciones a partir de su disponibilidad en plataformas específicas. No obstante, se implementaron acciones para asegurar \textbf{balance y diversidad}:

\begin{itemize}
    \item Igual número de canciones premiadas y no premiadas para evitar sesgo por clase.
    \item Inclusión de distintos géneros musicales (pop, reguetón, rock, balada, etc.).
    \item Distribución temporal amplia (años 90, 2000s, 2010s y 2020s).
\end{itemize}

Esta estrategia buscó garantizar la representatividad del fenómeno musical en español durante el periodo analizado, sin comprometer la calidad del análisis computacional posterior.
\section{Técnicas de Análisis de Datos}

Se aplicaron las siguientes técnicas:

\begin{itemize}
    \item \textbf{Procesamiento de Lenguaje Natural (PLN):} limpieza, lematización y vectorización con TF-IDF.
    \item \textbf{Aprendizaje supervisado:} uso de algoritmos de clasificación binaria.
    \item \textbf{Evaluación estadística:} cálculo de métricas de desempeño del modelo (accuracy, precision, recall, F1-score) y matrices de confusión.
    \item \textbf{Visualización:} nubes de palabras, gráficos de boxplot, curvas ROC.
\end{itemize}

\section{Identificación de Variables}

En la presente investigación se han identificado las siguientes variables clave, según su función dentro del diseño metodológico:

\newcolumntype{Y}{>{\raggedright\arraybackslash}X}
\begin{table}[H]
\centering
\caption{Identificación de Variables}
\begin{tabularx}{\textwidth}{|Y|Y|Y|Y|}
\hline
\textbf{Nombre de la variable} & \textbf{Tipo} & \textbf{Descripción} & \textbf{Escala de medición} \\
\hline
\textbf{Contenido lingüístico de las letras} & Independiente & Características léxicas y semánticas de las letras de canciones en español. & Ordinal (frecuencia, puntuaciones TF-IDF) \\
\hline
\textbf{Éxito musical (premiación)} & Dependiente & Condición de reconocimiento oficial otorgado por la RIAA a las canciones. & Nominal dicotómica (Premiada = 1, No premiada = 0) \\
\hline
\textbf{Desempeño del modelo clasificatorio} & Auxiliar & Capacidad del modelo para predecir correctamente si una canción es premiada o no. & De razón (porcentaje de F1-score, accuracy) \\
\hline
\end{tabularx}
\label{tab:identificacion_variables}
\end{table}

\section{Operacionalización de Variables}

La siguiente tabla resume las variables principales consideradas en la presente investigación, su tipo, dimensiones e indicadores, así como los instrumentos utilizados para su medición y análisis.

\newcolumntype{Y}{>{\raggedright\arraybackslash}X}

\begin{table}[H]
\centering
\caption{Tabla de Operacionalización de Variables}
\begin{tabularx}{\textwidth}{|Y|Y|Y|Y|Y|}
\hline
\textbf{Variable} & \textbf{Definición conceptual} & \textbf{Dimensiones} & \textbf{Indicadores} & \textbf{Instrumento} \\
\hline
\textbf{Contenido lingüístico de las letras (Independiente)} & Conjunto de elementos léxicos y semánticos presentes en letras de canciones en español. & Nivel léxico & Frecuencia de palabras, n-gramas, diversidad léxica & Script de procesamiento (Python + spaCy) \\
\cline{3-5}
& & Nivel semántico & Palabras afectivas, temas líricos, lematización & Modelo semántico, Wordcloud \\
\hline
\textbf{Éxito musical (premiación) (Dependiente)} & Reconocimiento institucional otorgado por la RIAA. & Clase de premiación & Premiación (1/0), tipo de premio & Base de datos RIAA \\
\hline
\textbf{Desempeño del modelo (Auxiliar)} & Nivel de acierto del modelo de clasificación supervisada. & Precisión predictiva & F1-score, accuracy, matriz de confusión & Scikit-learn, validación cruzada \\
\hline
\end{tabularx}
\label{tab:operacionalizacion}
\end{table}


\section{Tabla de Consistencia}

A continuación se presenta la matriz de consistencia metodológica que vincula los componentes fundamentales del estudio:

\begin{figure}[ht]
  \centering
  \includegraphics[width=\textwidth]{02_matriz_consistencia.png}
  \caption{Matriz de consistencia}
  \label{fig:matriz_consistencia}
\end{figure}

\section{Etapas del Estudio}

\begin{enumerate}
    \item Revisión documental del problema y casos similares.
    \item Recolección y depuración de datos textuales.
    \item Preprocesamiento con herramientas de PLN.
    \item Entrenamiento y validación cruzada de modelos supervisados.
    \item Análisis comparativo de métricas.
    \item Interpretación de resultados y discusión.
\end{enumerate}

\section{Cronograma}

En la tabla~\ref{tab:cronograma} se describen las fases del proyecto según CRISP-DM, junto con las actividades clave y sus fechas de inicio y fin. El cronograma abarca un año, desde julio de 2025 hasta junio de 2026.

\begin{table}[htbp]
  \centering
  \caption{Cronograma de actividades y fases CRISP-DM}
  \label{tab:cronograma}
  \begin{tabular}{@{}p{7cm}cc@{}}
    \toprule
    \textbf{Actividad}                           & \textbf{Inicio}      & \textbf{Fin}         \\
    \midrule
    \multicolumn{3}{@{}l}{\textbf{Fases CRISP-DM}} \\[0.2em]
    Comprensión del negocio                      & Julio 2025           & Agosto 2025          \\
    Comprensión de datos                         & Septiembre 2025      & Octubre 2025         \\
    Preparación de datos                         & Noviembre 2025       & Diciembre 2025       \\
    Modelado                                     & Enero 2026           & Febrero 2026         \\
    Evaluación                                   & Marzo 2026           & Abril 2026           \\
    Despliegue                                   & Mayo 2026            & Junio 2026           \\
    \addlinespace
    \multicolumn{3}{@{}l}{\textbf{Otras actividades}} \\[0.2em]
    Revisión bibliográfica                       & Julio 2025           & Agosto 2025          \\
    Diseño de instrumentos                       & Septiembre 2025      & Octubre 2025         \\
    Trabajo de campo / recolección de datos      & Noviembre 2025       & Diciembre 2025       \\
    Análisis de datos                            & Enero 2026           & Febrero 2026         \\
    Redacción de resultados                      & Marzo 2026           & Abril 2026           \\
    Revisión y corrección final                  & Mayo 2026            & Junio 2026           \\
    \bottomrule
  \end{tabular}
\end{table}


\section{Presupuesto}

\subsection{Recursos informáticos}

Para el desarrollo del proyecto se requerirá un entorno local basado en Ubuntu 24.04, sin necesidad de GPU, aprovechando hardware de gama media. El equipamiento previsto es el siguiente:

\begin{table}[H]
  \centering
  \caption{Características de hardware y software}
  \label{tab:recursos_informaticos}
  \begin{tabular}{|p{8cm}|p{8cm}|}
    \hline
    \textbf{Hardware}                         & \textbf{Software}               \\
    \hline
    Procesador: Intel Core i7-4700MQ          & Sistema operativo: Ubuntu 24.04 \\
    Memoria RAM: 16 GB                        & Python: v3.12.3                 \\
    Almacenamiento: 1 TB SSD (800 GB libre)   & Entorno virtual: Sí             \\
    Tarjeta madre: HP Envy 1968               & Jupyter Notebook: v8.1.0        \\
    \hline
  \end{tabular}
\end{table}

\subsection{Presupuesto estimado}

Dado que todo el software empleado será de código abierto y no requerirá licencias, el presupuesto se centrará en honorarios de expertos para la validación de resultados y en el equipo de cómputo en caso de no tenerse:

\begin{table}[H]
  \centering
  \caption{Presupuesto estimado}
  \label{tab:presupuesto_estimated}
  \begin{tabular}{|p{12cm}|r|}
    \hline
    \textbf{Concepto}                                                    & \textbf{Costo (Bs)} \\
    \hline
    Honorarios para expertos (encuesta de validación)                    & 1\,000              \\
    \hline
    Equipo de cómputo (Intel i7, 16 GB RAM, SSD o laptop equivalente)     & 8\,000              \\
    \hline
    \textbf{Total}                                                       & \textbf{9\,000}     \\
    \hline
  \end{tabular}
\end{table}
