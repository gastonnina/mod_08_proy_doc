\newpage
\begin{center}
  \textbf{Resumen Ejecutivo}
\end{center}

\noindent

El presente proyecto desarrolla un modelo supervisado para predecir la adopción de herramientas de inteligencia artificial en la comunidad de desarrolladores, utilizando datos de la encuesta global \textit{Stack Overflow Developer Survey 2025}. El proceso siguió una metodología completa que incluyó limpieza de datos, análisis exploratorio, selección de técnica analítica, evaluación de modelos competidores y optimización de hiperparámetros.

Los análisis iniciales mostraron que factores profesionales, tecnológicos y geográficos influyen de manera combinada en el uso de IA, aunque con correlaciones de baja magnitud. Con base en ello, se implementaron y compararon dos modelos principales: Regresión Logística y Random Forest, en versiones balanceadas y no balanceadas. El criterio de selección fue el \textit{F1-score}, dada la ligera desproporción entre clases.

El modelo con mejor desempeño y mayor estabilidad fue la \textbf{Regresión Logística sin balance}, que obtuvo un F1 promedio de $0.8817$ con baja variabilidad. Tras aplicar técnicas de optimización (GridSearchCV y RandomizedSearchCV), se logró un F1 cercano a $0.883$, además de un aumento sustancial en el \textit{recall} de la clase positiva, reduciendo los falsos negativos de $105$ a $17$. Si bien la mejora no alcanzó el 5\% sugerido, la robustez y explicabilidad del modelo lo convierten en la mejor alternativa para cumplir los objetivos del estudio.

Los resultados permiten identificar patrones de adopción de IA en desarrolladores globales y proveer información útil para estrategias de formación, talento y adopción tecnológica en organizaciones.

\section*{Enlaces del Proyecto}

\textbf{Sitio Web del Sprint 1 (SITIO WEB):}\\
\url{https://gastonnina.github.io/miadas_mod_08_proy/}

\vspace{0.5cm}

\textbf{Repositorio del Proyecto: (NOTEBOOK y CODIGO SITIO WEB)}\\
\url{https://github.com/gastonnina/miadas_mod_08_proy}

\vspace{1cm}