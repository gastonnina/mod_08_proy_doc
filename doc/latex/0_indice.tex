\newpage
\fancypagestyle{plain}{
	\fancyhf{}
	\fancyhead[R]{\thepage}
	\renewcommand{\headrulewidth}{0pt}
}
\renewcommand{\cftchapdotsep}{\cftdotsep} % activa puntos suspensivos para capítulos
\renewcommand{\contentsname}{ÍNDICE}
% Cambia el tamaño del título del índice
\renewcommand{\cfttoctitlefont}{\large\bfseries}  % o \Large, \normalsize, etc.
\renewcommand{\cftloftitlefont}{\large\bfseries}
\renewcommand{\cftlottitlefont}{\large\bfseries}

\tableofcontents

\cleardoublepage   % O \newpage
\phantomsection    % Asegura que se enlace bien si usas hyperref
%\addcontentsline{toc}{chapter}{Índice de Figuras}  % Agrega entrada al índice general
\renewcommand{\listfigurename}{\MakeUppercase{Índice de Figuras}}
\listoffigures


\cleardoublepage   % O \newpage
\phantomsection    % Asegura que se enlace bien si usas hyperref
\renewcommand{\tablename}{Tabla}
\renewcommand{\listtablename}{\MakeUppercase{Índice de Tablas}}
\listoftables

% \cleardoublepage   % O \newpage
% \phantomsection    % Asegura que se enlace bien si usas hyperref
% \renewcommand{\lstlistingname}{Código}               % Para los captions (singular)
% \renewcommand{\lstlistlistingname}{\MakeUppercase{Índice de Códigos}} % Para el índice
% \lstlistoflistings

