\newpage
\thispagestyle{empty}
\vspace*{0.35\textheight}
\begin{center}
	{\Huge\textbf{CAPÍTULO VI}} \\[0.5cm]
	{\Huge\textbf{CONCLUSIONES Y}}\\[0.5cm]
    {\Huge\textbf{RECOMENDACIONES}}
\end{center}

\newpage
\chapter{CONCLUSIONES Y RECOMENDACIONES}

\section{Conclusiones}

El presente proyecto logró construir y evaluar un modelo supervisado capaz de predecir la adopción de herramientas de inteligencia artificial entre desarrolladores, utilizando datos de la encuesta global \textit{Stack Overflow Developer Survey 2025}. A través de un proceso metodológico completo —comprendiendo exploración, selección de variables, comparación de modelos y optimización— se obtuvieron resultados consistentes y alineados con el criterio de éxito definido en el Sprint.

En primer lugar, el análisis exploratorio mostró que la adopción de IA no depende de un único factor dominante, sino de combinaciones de características profesionales, tecnológicas y contextuales. Variables como \textbf{DevType}, \textbf{Country}, \textbf{Industry} y \textbf{OrgSize} presentaron correlaciones moderadas con el uso de IA, mientras que \textbf{WorkExp} y \textbf{EdLevel} mostraron contribuciones más leves. Esta estructura multifactorial justificó el uso de un modelo supervisado capaz de manejar tanto variables categóricas como numéricas.

La comparación de técnicas predictivas permitió evaluar el desempeño de \textbf{Regresión Logística} y \textbf{Random Forest} en versiones balanceadas y no balanceadas. Ambos modelos superaron el criterio de éxito (F1 $\geq 0.80$), pero la \textbf{Regresión Logística sin balance} destacó por su rendimiento competitivo (F1 $\approx0.8817$), su baja variabilidad entre folds (0.0015) y su interpretabilidad, factores clave para análisis explicativos y toma de decisiones.

El proceso de optimización confirmó que el modelo baseline ya capturaba la mayor parte de la estructura del problema. La búsqueda exhaustiva con \textit{GridSearchCV} y la búsqueda aleatoria con \textit{RandomizedSearchCV} convergieron en hiperparámetros similares, alcanzando un F1 promedio de $\approx0.883$. Si bien la mejora respecto al modelo base fue modesta, el ajuste refinó significativamente el \textbf{recall} de la clase positiva, reduciendo falsos negativos de forma sustancial (105→17), lo que resulta valioso para escenarios donde omitir usuarios de IA representa un mayor riesgo que clasificarlos erróneamente como tales.

En términos generales, el modelo final ofrece un equilibrio adecuado entre rendimiento, estabilidad, eficiencia computacional y capacidad de interpretación. No obstante, persisten limitaciones propias del dataset, como la naturaleza autodeclarada de la encuesta y la variabilidad cultural asociada a los países participantes. Asimismo, el AUC moderado ($\approx0.67$) indica que aún existe superposición entre clases, lo que podría explorarse en trabajos futuros mediante modelos más complejos o nuevas fuentes de información.

En síntesis, se concluye que la \textbf{Regresión Logística optimizada} constituye una solución sólida y confiable para predecir la adopción de IA con las variables disponibles. El modelo proporciona insights relevantes para comprender qué perfiles profesionales, regiones e industrias presentan mayor propensión a integrar herramientas de IA, lo cual puede orientar estrategias de formación, políticas de talento y decisiones organizacionales en un contexto tecnológico en rápida evolución.

\section{Limitaciones}

A pesar de los resultados obtenidos, el presente estudio presenta varias limitaciones que deben considerarse al interpretar las conclusiones:

\begin{itemize}
    \item \textbf{Naturaleza autodeclarada del dataset.} La encuesta Stack Overflow 2025 se basa en respuestas voluntarias, lo que introduce sesgos de percepción, autoselección y representatividad desigual entre países.

    \item \textbf{Variables limitadas respecto al dataset original.} El análisis utilizó un subconjunto reducido de variables, lo que puede omitir factores relevantes asociados al uso de IA (por ejemplo, exposición a herramientas específicas, motivaciones personales o cultura organizacional).

    \item \textbf{Correlaciones bajas entre predictores y la variable objetivo.} La mayoría de las asociaciones cuantitativas fueron débiles, lo que reduce la capacidad teórica de separación entre clases y explica el AUC moderado ($\approx 0.67$).

    \item \textbf{Posible ruido en variables categóricas de alta cardinalidad.} Categorías como \textit{DevType} o \textit{Country} pueden presentar ruido semántico o dispersión significativa, afectando la estabilidad del modelo.

    \item \textbf{Ausencia de análisis temporal.} El estudio es transversal y no permite evaluar la evolución del uso de IA a lo largo del tiempo.
\end{itemize}

Estas limitaciones no invalidan los resultados, pero sí sugieren cautela al generalizar los hallazgos y señalan oportunidades para mejorar el modelo en trabajos futuros.

\section{Recomendaciones y Trabajo Futuro}

Con base en los resultados del modelo final y en las limitaciones identificadas, se proponen las siguientes recomendaciones para estudios posteriores y aplicaciones prácticas:

\subsection*{Recomendaciones para organizaciones o tomadores de decisiones}

\begin{itemize}
    \item \textbf{Enfocar programas de formación en perfiles técnicos clave.} Roles como científicos de datos, ingenieros de software y desarrolladores especializados muestran mayor propensión a adoptar IA; fortalecer su capacitación puede acelerar procesos de innovación.

    \item \textbf{Priorizar estrategias de adopción según industria y región.} Sectores como software, telecomunicaciones y fintech presentan mayor afinidad con herramientas de IA, mientras que otros requieren políticas más específicas.

    \item \textbf{Utilizar modelos explicables para toma de decisiones.} La Regresión Logística permite identificar qué factores influyen en la adopción, facilitando planes de formación y políticas organizacionales basadas en evidencia.
\end{itemize}

\subsection*{Trabajo futuro}

\begin{itemize}
    \item \textbf{Incorporar nuevas variables del dataset completo}, como tecnologías específicas utilizadas, tipo de proyecto, motivaciones personales y nivel de exposición a IA.

    \item \textbf{Evaluar modelos más complejos}, tales como XGBoost, LightGBM o redes neuronales, para explorar si pueden mejorar el AUC y la capacidad de discriminación.

    \item \textbf{Experimentar con métodos de \textit{feature engineering}}: extracción de tópicos en descripciones, codificación semántica de roles o embeddings de países e industrias.

    \item \textbf{Realizar análisis temporal} con ediciones anteriores y futuras de la encuesta para identificar tendencias en adopción de IA a lo largo del tiempo.

    \item \textbf{Explorar técnicas de calibración de probabilidades}, como Platt scaling o isotonic regression, para mejorar la interpretación probabilística del modelo.

    \item \textbf{Implementar el modelo en un entorno productivo} (API o dashboard) para monitorear la adopción de IA en diversas poblaciones o empresas.
\end{itemize}

Estas recomendaciones buscan fortalecer la capacidad predictiva del modelo y ampliar su aplicación práctica en contextos reales de toma de decisiones.
