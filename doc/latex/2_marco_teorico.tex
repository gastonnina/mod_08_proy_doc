\newpage
\thispagestyle{empty}
\vspace*{0.35\textheight}
\begin{center}
	{\Huge\textbf{CAPÍTULO III}} \\[0.5cm]
	{\Huge\textbf{DESCRIPCIÓN DEL CONJUNTO DE DATOS}}
\end{center}

\newpage
\chapter{DESCRIPCIÓN DEL CONJUNTO DE DATOS}

\section{Origen y Contexto del Dataset}

El análisis se desarrolla utilizando el \textit{Stack Overflow Annual Developer Survey 2025}, publicado en la plataforma Kaggle \parencite{kaggle2025}. El dataset original contiene \textbf{49\,123 filas} y \textbf{170 columnas}, incluyendo preguntas de respuesta múltiple, variables categóricas y campos abiertos.

\begin{center}
\url{https://www.kaggle.com/datasets/edoardogalli/stack-overflow-annual-developer-survey-2025}
\end{center}

Este conjunto de datos reúne respuestas de desarrolladores de todo el mundo y contiene información relevante sobre su formación, experiencia, tecnologías utilizadas, modalidad de trabajo y el uso actual de herramientas de inteligencia artificial. Su amplitud y enfoque en perfiles tecnológicos lo convierten en una fuente adecuada para estudiar factores asociados a la adopción de IA.

\section{Variables Incluidas}

Para este estudio se seleccionó un subconjunto de variables que presentan relevancia predictiva y consistencia conceptual respecto a la pregunta de investigación. La Tabla \ref{tab:variables_dataset} resume las variables empleadas:

\begin{table}[H]
\centering
\caption{Variables seleccionadas del dataset Stack Overflow 2025}
\label{tab:variables_dataset}
\begin{tabular}{p{3.5cm} p{10cm}}
\toprule
\textbf{Variable} & \textbf{Descripción} \\
\midrule
\texttt{AI\_Usage} & Variable objetivo (1 = usa IA, 0 = no usa IA). \\
\texttt{DevType} & Rol profesional del desarrollador. \\
\texttt{WorkExp} & Años de experiencia laboral. \\
\texttt{NumLanguages} & Recuento numérico derivado de la columna original LanguageHaveWorkedWith. \\
\texttt{Country} & País de residencia. \\
\texttt{RemoteWork} & Modalidad de trabajo (remoto, híbrido o presencial). \\
\texttt{Industry} & Industria o sector en el que trabaja el encuestado. \\
\texttt{OrgSize} & Tamaño de la organización. \\
\texttt{EdLevel} & Nivel de formación académica alcanzado. \\
\bottomrule
\end{tabular}
\end{table}

Estas variables combinan atributos categóricos y numéricos, lo que permite aplicar modelos de clasificación supervisada mediante estrategias de codificación y escalamiento.

\section{Variables Predictoras Utilizadas}

Las variables seleccionadas para el modelamiento se organizan por tipo conceptual, tal como se muestra en la Tabla \ref{tab:predictoras_usadas}. Esta clasificación facilita la interpretación del modelo y asegura coherencia con la literatura sobre adopción tecnológica.

\begin{table}[H]
\centering
\caption{Predictoras utilizadas y justificación conceptual}
\label{tab:predictoras_usadas}
\begin{tabular}{p{3.5cm} p{4cm} p{7cm}}
\toprule
\textbf{Tipo de variable} & \textbf{Columna} & \textbf{Justificación} \\
\midrule
Profesional     & \texttt{DevType}       & El rol del desarrollador influye en la adopción de IA. \\
Experiencia     & \texttt{WorkExp}       & Años de experiencia laboral. \\
Tecnológica     & \texttt{LanguageHaveWorkedWith} & Lenguajes utilizados (Python, R, etc.). \\
Geográfica      & \texttt{Country}       & Contexto regional y acceso a IA. \\
Laboral         & \texttt{RemoteWork}    & Modalidad de trabajo. \\
Sectorial       & \texttt{Industry}      & Tipo de industria o empresa. \\
Organizacional  & \texttt{OrgSize}       & Tamaño de la organización. \\
Educativa       & \texttt{EdLevel}       & Nivel de formación académica. \\
\bottomrule
\end{tabular}
\end{table}

\section{Procesamiento y Limpieza de Datos}

El dataset original descargado desde Kaggle contiene \textbf{49\,123 filas} y \textbf{170 columnas}, incluyendo múltiples variables textuales de respuesta múltiple, así como valores faltantes en distintas secciones del formulario. El primer paso del preprocesamiento consistió en evaluar la estructura global del dataset y la distribución de valores nulos.


\begin{itemize}
    \item Eliminación de filas con valores faltantes en la variable objetivo \texttt{AI\_Usage}.
		\item Imputación de valores faltantes en \texttt{WorkExp} mediante la mediana.
    \item Imputación de valores faltantes en variables categóricas usando la categoría genérica \textit{``No especificado''}.
    \item Normalización del formato de variables categóricas.
    \item Conversión de la columna \texttt{LanguageHaveWorkedWith} en un recuento numérico denominado \texttt{NumLanguages}.
    \item Estandarización de variables numéricas mediante \textit{StandardScaler}.
    \item Codificación \textit{One-Hot Encoding} para todas las variables categóricas.
\end{itemize}

Estas transformaciones son consistentes con buenas prácticas de preprocesamiento para modelos de clasificación.

\subsection{Inspección de Valores Faltantes}

La Figura \ref{fig:missing_matrix} presenta la matriz de valores faltantes generada a partir de
las variables seleccionadas para el estudio. Esta visualización permite identificar patrones
sistemáticos de ausencia de datos y guiar las decisiones de imputación o eliminación.

\begin{figure}[H]
\centering
\includegraphics[width=\textwidth]{images/02_valores_faltantes.png}
\caption{Matriz de valores faltantes en las variables predictoras seleccionadas.}
\label{fig:missing_matrix}
\end{figure}

Tras esta inspección se identificó que algunas columnas contenían valores faltantes esporádicos, especialmente en \texttt{DevType}, \texttt{Industry} y \texttt{RemoteWork}. Sin embargo, la variable objetivo derivada (\texttt{AI\_Usage}) debía estar completamente observada para mantener la coherencia del enfoque supervisado. Por ello se decidió:

\begin{itemize}
    \item Eliminar filas con valores faltantes en la variable objetivo.
    \item Conservar filas con valores faltantes en columnas predictoras, dado que su proporción
    era baja y podían manejarse mediante codificación categórica.
\end{itemize}

Después del filtrado inicial, el dataset quedó reducido a \textbf{33\,686 filas} y \textbf{9 columnas} (selección final de predictoras + variable objetivo).

\subsection{Construcción de la Variable Objetivo \texttt{AI\_Usage}}

La variable objetivo \texttt{AI\_Usage} fue construida a partir de la pregunta original del formulario: \textit{``Do you currently use AI tools in your development process?''}. Esta pregunta incluye cinco opciones que representan distintos niveles de adopción de herramientas de inteligencia artificial:

\begin{itemize}
    \item Yes, I use AI tools daily.
    \item Yes, I use AI tools weekly.
    \item Yes, I use AI tools monthly or infrequently.
    \item No, but I plan to soon.
    \item No, and I don't plan to.
\end{itemize}

Para los fines del presente estudio, estas categorías fueron agrupadas con el objetivo de crear una variable binaria adecuada para técnicas de clasificación supervisada. El criterio de binarización fue el siguiente:

\begin{itemize}
    \item \textbf{AI\_Usage = 1}: participantes que declararon utilizar herramientas de IA de manera diaria, semanal o mensual.
    \item \textbf{AI\_Usage = 0}: participantes que no utilizan IA o no planean utilizarla.
\end{itemize}

Esta transformación permite convertir una pregunta de escala múltiple en una variable objetivo consistente y directamente alineada con la pregunta de investigación: predecir si un desarrollador utiliza o no herramientas de inteligencia artificial.

Esta variable fue verificada para asegurar que no quedaran valores faltantes tras la binarización y que las proporciones de clases fueran adecuadas para el modelamiento.



\section{Justificación de Exclusiones y Transformaciones}

Se excluyeron variables redundantes o con granularidad excesiva (como listas textuales extremamente largas) para asegurar un procesamiento eficiente y evitar problemas de dimensionalidad. La variable objetivo \texttt{AI\_Usage} se mantuvo como binaria porque responde directamente a la pregunta planteada y permite implementar un enfoque predictivo sin ambigüedad conceptual.

Asimismo, la transformación de lenguajes de programación a un recuento (\texttt{NumLanguages}) se justifica por su contribución a la interpretabilidad del modelo y la reducción de ruido.

\section{Visualización Inicial del Dataset}

Se realizaron gráficos exploratorios univariados y bivariados para identificar patrones iniciales y relaciones potenciales entre las variables predictoras y la variable objetivo. Estos gráficos incluyen distribuciones de experiencia laboral, relación entre sector e \texttt{AI\_Usage}, modalidad de trabajo y correlación mixta. Las visualizaciones completas se presentan en el capítulo correspondiente al análisis exploratorio.