\newpage
\thispagestyle{empty}
\vspace*{0.35\textheight}
\begin{center}
	{\Huge\textbf{CAPÍTULO I}} \\[0.5cm]
	{\Huge\textbf{MARCO INTRODUCTORIO}}
\end{center}
\newpage
\thispagestyle{fancy}

\newpage
\pagenumbering{arabic}
\section*{}
\begin{center}
	\Large \textbf{INTRODUCCIÓN}
\end{center}

En los últimos años, la adopción de herramientas de Inteligencia Artificial (IA) en el  ámbito del desarrollo de software ha experimentado un crecimiento acelerado. La aparición de modelos generativos, asistentes de código y plataformas de automatización ha  transformado las dinámicas de trabajo de los desarrolladores, influyendo en productividad, competencias técnicas y procesos de toma de decisiones dentro de las organizaciones  tecnológicas.

El dataset \textit{Stack Overflow Annual Developer Survey 2025} constituye una de las  fuentes más completas y actuales sobre el perfil profesional, tecnológico y educativo de  desarrolladores a nivel global. Este conjunto de datos permite explorar de manera cuantitativa  los factores asociados a la adopción de herramientas de IA y genera un marco idóneo para  construir modelos predictivos basados en aprendizaje automático.

El propósito general de este estudio es desarrollar un modelo supervisado capaz de  predecir si un encuestado utiliza herramientas de IA, empleando variables profesionales, educativas y tecnológicas. Para ello, se realiza un proceso analítico que incluye limpieza de  datos, análisis exploratorio, selección de técnicas, comparación de modelos y validación  cuantitativa utilizando métricas de desempeño como F1-score y curvas ROC.

Metodológicamente, se aplican dos enfoques de clasificación ampliamente utilizados:  regresión logística y Random Forest. Ambos modelos se evalúan mediante validación cruzada  estratificada (k=5), considerando la naturaleza binaria de la variable objetivo y el leve  desbalance presente en el conjunto de datos. Asimismo, se analizan las ventajas, limitaciones  y estabilidad de cada modelo en función de su desempeño y su interpretabilidad.

Finalmente, este informe se estructura de la siguiente manera: en el Capítulo 2 se presenta  el marco teórico relacionado con el modelamiento supervisado y las métricas utilizadas; en el  Capítulo 3 se describe el conjunto de datos y se desarrollan los análisis exploratorios; en el  Capítulo 4 se exponen los resultados del modelamiento y la comparación entre técnicas; y en  el Capítulo 5 se presentan las conclusiones, implicaciones prácticas y recomendaciones derivadas del estudio.


\chapter{MARCO INTRODUCTORIO}

\section{Planteamiento del Problema y Pregunta de Investigación}
\subsection{Contextualización del Problema}
El uso de herramientas basadas en Inteligencia Artificial (IA) se ha expandido de manera acelerada en el ámbito del desarrollo de software. Sin embargo, los niveles de adopción varían sustancialmente entre diferentes perfiles profesionales, regiones, industrias y niveles educativos. Esto plantea la necesidad de identificar los factores que explican esta variabilidad y permiten anticipar patrones de adopción tecnológica.

El dataset público \textit{Stack Overflow Annual Developer Survey 2025} ofrece información estructurada sobre miles de desarrolladores a nivel global, permitiendo abordar este problema desde un enfoque cuantitativo y predictivo.

\subsection{Problema de Investigación}
A pesar de la disponibilidad de datos, no se conoce con claridad qué características del perfil de un desarrollador influyen más en la probabilidad de que utilice o no herramientas de IA. Tampoco se ha establecido un modelo sistemático que permita predecir este comportamiento y evaluar su desempeño mediante métricas apropiadas.

\subsection{Pregunta de Investigación Principal}

La pregunta central que guía el presente estudio es la siguiente:

\begin{quote}
\textbf{¿Podemos predecir la adopción de IA en desarrolladores utilizando su perfil profesional y tecnológico en 2025?}
\end{quote}

Esta formulación permite estructurar el análisis en torno a una variable objetivo claramente definida (\textit{AI\_Usage}) y a un conjunto de variables predictoras que combinan atributos categóricos y numéricos relevantes para el comportamiento tecnológico.

\subsection{Tipo de Pregunta Analítica}

De acuerdo con la naturaleza de la variable objetivo (\textit{usa IA} = Sí/No), el estudio se enmarca en una \textbf{pregunta analítica de tipo predictivo}. Esto implica que el propósito no es explicar causalmente por qué se adopta la IA, sino construir un modelo capaz de estimar la probabilidad de adopción utilizando patrones aprendidos a partir de los datos disponibles.

El problema se aborda mediante técnicas de \textbf{clasificación supervisada}, ya que el dataset incluye etiquetas reales que permiten entrenar modelos predictivos bajo un esquema de aprendizaje supervisado.

La pregunta planteada corresponde a un análisis de tipo:

\begin{itemize}
    \item \textbf{Predictivo}: se busca estimar un resultado futuro o no observado.
    \item \textbf{Supervisado}: la variable objetivo (\texttt{AI\_Usage}) está disponible y es binaria.
    \item \textbf{De clasificación}: el análisis requiere asignar una categoría (usa IA / no usa IA).
\end{itemize}

\subsection{Justificación del Enfoque de Modelamiento}
La elección de un enfoque supervisado de clasificación se justifica por las siguientes razones:

\begin{enumerate}
    \item El dataset contiene una \textbf{variable objetivo claramente definida} relacionada con el uso de herramientas de IA.
    \item La naturaleza de la variable objetivo (binaria) permite la aplicación de modelos como regresión logística y Random Forest.
    \item Existe interés práctico en \textbf{predecir la probabilidad de adopción} como apoyo a decisiones de formación, estrategias de talento y políticas internas en organizaciones tecnológicas.
    \item Las características de los desarrolladores (rol, país, industria, experiencia, educación) pueden representar factores relevantes y medibles para la predicción.
\end{enumerate}

\subsection{Objetivo}

Predecir la probabilidad de que un encuestado utilice herramientas de inteligencia artificial (\textit{AI Usage}: Sí/No) en función de sus características técnicas y demográficas.

\section{Hipótesis}

Un modelo predictivo permite identificar la tendencia de uso de IA en el desarrollo de software

\section{Criterio de Éxito}

El proyecto se considera exitoso si el modelo predictivo cumple con el siguiente criterio:

\begin{quote}
\textbf{F1-score mayor o igual a 0.80 en el conjunto de prueba, utilizando validación cruzada
con $k = 5$ para asegurar la estabilidad del desempeño.}
\end{quote}
