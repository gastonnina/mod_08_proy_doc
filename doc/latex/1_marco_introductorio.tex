\newpage
\thispagestyle{empty}
\vspace*{0.35\textheight}
\begin{center}
	{\Huge\textbf{CAPÍTULO I}} \\[0.5cm]
	{\Huge\textbf{MARCO INTRODUCTORIO}}
\end{center}
\newpage
\thispagestyle{fancy}

\newpage
\pagenumbering{arabic}
\section*{}
\begin{center}
	\Large \textbf{INTRODUCCIÓN}
\end{center}

En la era digital, el análisis de texto ha cobrado gran relevancia en diversos dominios, incluyendo el ámbito musical, donde las letras de canciones representan una fuente rica en información lingüística y cultural. Con el auge de las técnicas de Procesamiento del Lenguaje Natural (PLN) y aprendizaje automático, es posible explorar patrones en estos datos textuales de gran volumen, con el fin de descubrir tendencias, clasificar contenidos e incluso predecir el éxito de productos culturales como la música a escala mundial \parencite{Oancea2023,Hassan2022}.

Numerosos estudios han demostrado que las letras de canciones pueden ser analizadas de forma efectiva mediante modelos supervisados, utilizando representaciones vectoriales como \textit{TF-IDF} para transformar el texto en una forma numérica adecuada para algoritmos de clasificación \parencite{Wardana2024}. Estos enfoques han sido aplicados, por ejemplo, para detectar contenido ofensivo en canciones \parencite{Bolla2024}, realizar análisis de sentimiento en distintos idiomas \parencite{Kumar2024}, y agrupar letras por temática, género o estado de ánimo mediante técnicas de agrupamiento basadas únicamente en el contenido textual \parencite{Parra2013}.

Sin embargo, no existe aún un modelo computacional que identifique objetivamente los patrones lingüísticos que diferencian las letras de canciones en español premiadas por la Recording Industry Association of America (RIAA) de las no premiadas. Esta tesis propone desarrollar, sobre un corpus mundial de letras de canciones etiquetado (1990–2022) que garantiza delimitación temporal y cobertura global, y se desarrollará un modelo supervisado que incluya: limpieza y normalización del texto; vectorización \textit{TF-IDF} con n-gramas; y clasificación mediante algoritmos supervisados como Naive Bayes, Regresión Logística, Máquinas de Vectores de Soporte (SVM), Random Forest y XGBoost.

Para evaluar el desempeño del modelo, se emplearán métricas como precisión, \textit{recall} y \textit{F1-score}, con el objetivo de aportar evidencia empírica al estudio del éxito musical desde una perspectiva computacional y lingüística. Así, se abren nuevas posibilidades para la industria cultural y musical, así como para el diseño de sistemas de recomendación inteligentes que integren características estilísticas y lingüísticas de las letras de canciones.

\chapter{MARCO INTRODUCTORIO}

\section{Antecedentes}

El estudio de las letras de canciones ha evolucionado a partir de los fundamentos del Procesamiento del Lenguaje Natural establecidos por Jurafsky \& Martin \parencite{Jurafsky2008}, y actualmente existen repositorios como Genius y Letras.com que permiten recolectar grandes volúmenes de letras en español \parencite{GeniusAPI2022}, al mismo tiempo que la RIAA publica listados históricos de premiaciones que facilitan el análisis de la relación entre características textuales y reconocimiento institucional \parencite{RIAAstats2023}.



\section{Planteamiento del problema}

La mayoría de los estudios sobre éxito musical se basan en indicadores cuantitativos ventas, reproducciones, rankings y en percepciones subjetivas tales como opiniones de críticos, redes sociales, dejando de lado el análisis sistemático del contenido textual de las letras \parencite{Hassan2022,Wardana2024}. En particular, no existe un modelo computacional que, de forma objetiva, identifique patrones lingüísticos en canciones en español premiadas por la RIAA frente a las no premiadas. Esta brecha limita el desarrollo de herramientas que integren características lingüísticas para comprender y predecir el reconocimiento institucional de obras musicales.


\section{Formulación del problema de investigación}

\vspace{1em}
\noindent
\textit{¿Qué patrones lingüísticos diferencian las letras de canciones en español que han sido premiadas por la RIAA de aquellas que no han sido premiadas durante el período 1990--2022, y cómo pueden ser identificados mediante un modelo de clasificación supervisado?}
\vspace{1em}

\section{Delimitación}
El estudio se centra en un corpus mundial de letras de canciones en español publicadas entre 1990 y 2022, diferenciando aquellas premiadas por la RIAA de las no premiadas.

\section{Problemas específicos}

A partir de esta pregunta principal, se derivan los siguientes problemas específicos:

\begin{itemize}
    \item ¿Cómo se puede construir un corpus representativo y balanceado de letras premiadas y no premiadas?
    \item ¿Qué técnicas de procesamiento de lenguaje natural son más adecuadas para representar lingüísticamente las letras?
    \item ¿Qué algoritmos supervisados presentan mejor rendimiento en la clasificación de canciones según su premiación?
    \item ¿Cuáles son los rasgos lingüísticos más discriminativos entre las canciones premiadas y no premiadas?
    \item ¿Cómo representar visualmente los patrones lingüísticos identificados por los modelos?
\end{itemize}

\section{Objetivos de la investigación}
    \subsection{Objetivo general}

    \noindent
    Desarrollar un modelo de clasificación supervisado basado en técnicas de Procesamiento de Lenguaje Natural y algoritmos de machine learning que permita identificar y analizar patrones lingüísticos diferenciadores en letras de canciones en español premiadas y no premiadas por la Recording Industry Association of America (RIAA) en el periodo 1990-2022, evaluando su desempeño mediante precisión, \textit{recall} y \textit{F1‐score}.

    \subsection{Objetivos específicos}

    \begin{itemize}
        \item Recolectar un corpus balanceado de letras de canciones en español premiadas y no premiadas por la RIAA (1990–2022).
        \item Preprocesar el texto mediante limpieza, normalización, eliminación de \emph{stopwords}, lematización y extracción de n-gramas, y vectorizar con TF-IDF.
        \item Entrenar modelos supervisados (Naive Bayes, Regresión Logística, SVM, Random Forest y XGBoost) para clasificar las letras.
        \item Evaluar el desempeño de los modelos utilizando precisión, \textit{recall} y \textit{F1‐score}.
        \item Visualizar los patrones lingüísticos detectados mediante representaciones gráficas e informes interpretables.
    \end{itemize}

\section{Hipótesis}

\noindent
\textbf{Hipótesis principal:} \textit{Existen patrones lingüísticos diferenciables entre las letras de canciones en español que han sido premiadas por la RIAA y aquellas que no lo han sido, los cuales pueden ser identificados mediante un modelo de clasificación supervisado con un nivel de precisión estadísticamente significativo.}

\subsection{Hipótesis específicas}

A partir del objetivo general y de los cinco objetivos específicos, se plantean las siguientes hipótesis:

\begin{enumerate}
  \item \textbf{Corpus balanceado.}  
  El corpus recolectado presentará una proporción 1:1 de canciones premiadas y no premiadas por la RIAA, manteniendo una distribución representativa de géneros musicales y periodos temporales (1990–2022).

  \item \textbf{Preprocesamiento y vectorización.}  
  El uso combinado de limpieza textual, normalización, eliminación de stopwords, lematización y extracción de n-gramas generará vectores TF-IDF con mayor capacidad discriminativa (medida en varianza explicada en un análisis de componentes principales) que una simple vectorización unigram.

  \item \textbf{Comparación de modelos.}  
  El modelo XGBoost alcanzará un F1-score significativamente superior (p < 0.05) al de Naive Bayes en el conjunto de prueba, demostrando la ventaja del boosting en datos textuales con desequilibrio de clases.

  \item \textbf{Patrones lingüísticos.}  
  Las variables afectivas (palabras con carga emocional) y los n-gramas de orden dos y tres serán los indicadores más influyentes para distinguir letras premiadas de no premiadas, según la importancia de características extraída de Random Forest y Regresión Logística.

  \item \textbf{Visualización de resultados.}  
  Las nubes de palabras y los mapas de calor mostrarán agrupaciones temáticas claramente diferenciadas entre canciones premiadas y no premiadas, facilitando la interpretación de los patrones lingüísticos identificados.
\end{enumerate}

\section{Justificación}

\subsection{Justificación teórica}

Desde una perspectiva teórica, este estudio se fundamenta en los avances del Procesamiento del Lenguaje Natural (PLN) y de la clasificación supervisada como herramientas clave para el análisis de datos textuales \parencite{Jurafsky2008,Manning2014}. En particular, las letras de canciones pueden entenderse como documentos estructurados que, transformados en vectores mediante TF-IDF, revelan patrones lingüísticos significativos \parencite{Ramos2003}.

\subsection{Justificación práctica y social}

Esta investigación aporta un modelo capaz de clasificar el éxito de una canción basándose en su contenido lírico, lo cual resulta de gran utilidad para compositores, productores y plataformas de distribución musical que buscan comprender qué rasgos lingüísticos influyen en la recepción de una obra. Asimismo, el arte y la música desempeñan un rol central en la identidad cultural y emocional de la sociedad, y analizar el éxito desde una perspectiva lingüística facilita la comprensión de cómo el lenguaje en las letras conecta cognitivamente con los oyentes, influye en su percepción y refuerza los vínculos culturales. La metodología propuesta, que abarca desde la recolección y el preprocesamiento del texto hasta la aplicación de modelos supervisados y la interpretación de resultados, es replicable y puede adaptarse a otros idiomas o contextos culturales, lo que democratiza el uso de sistemas de recomendación y herramientas de análisis automatizado en la industria musical y en las ciencias sociales.

\subsection{Justificación metodológica}

La presente investigación adopta un enfoque mixto que combina técnicas de procesamiento de lenguaje natural (TF-IDF con n-gramas) con un ensamble de cinco algoritmos de clasificación (Naive Bayes, Regresión Logística, SVM, Random Forest y XGBoost), todo bajo el marco de la metodología CRISP-DM \parencite{Chapman2000}. Esta configuración permite no solo comparar la efectividad de cada modelo en la predicción del éxito musical a partir de un corpus mundial de letras, sino también integrar de forma sistemática las fases de comprensión de datos, modelado y evaluación, aportando una ruta reproducible y escalable para estudios de texto en español. Además, la operacionalización detallada de variables afectivas y lingüísticas introduce un nivel de granularidad poco explorado en estudios previos, lo que fortalece la robustez y aplicabilidad de los resultados.

\section{Alcances y limitaciones}

\subsection{Alcances}

\begin{itemize}
  \item Desarrollo de un modelo supervisado basado en vectorización TF-IDF para clasificar letras de canciones en español.
  \item Evaluación de cinco algoritmos supervisados: Naive Bayes, Regresión Logística, SVM, Random Forest y XGBoost.
  \item Análisis exclusivo del contenido textual de las canciones, sin considerar métricas de reproducción ni aspectos de producción.
  \item Aplicación al periodo 1990–2022 y canciones premiadas por la RIAA con disponibilidad de letra.
\end{itemize}

\subsection{Limitaciones}

\begin{itemize}
  \item Análisis limitado a letras de canciones en español con información disponible sobre premiaciones.
  \item Exclusión de letras traducidas y de composiciones instrumentales sin letra.
  \item Dependencia de la calidad y cobertura del conjunto de datos recopilado.
  \item Generalización potencialmente reducida: podría requerirse ajustar la metodología para otros géneros, idiomas o contextos culturales.
\end{itemize}


\section{Resultados esperados}

Se espera contar con un corpus balanceado de letras de canciones en español, etiquetado según su premiación por la RIAA en el periodo 1990–2022, y con un proceso reproducible de preprocesamiento y vectorización TF-IDF aplicado al mismo. Asimismo, se prevé la implementación y entrenamiento de cinco modelos supervisados (Naive Bayes, Regresión Logística, SVM, Random Forest y XGBoost) para la clasificación de las letras, cuya comparación de desempeño se realizará mediante precisión, \textit{recall} y \textit{F1-score}, y que permitirá identificar el modelo con mejor rendimiento. A partir de los modelos entrenados, se extraerán los patrones lingüísticos más discriminativos, a través de la importancia de variables de cada algoritmo, y se presentarán en un conjunto de visualizaciones e informes interpretables (gráficos de importancia de variables, nubes de términos y mapas de calor) que respalden las conclusiones y recomendaciones de la tesis.

\section{Matriz de Consistencia}

A continuación se presenta la matriz de consistencia, en la que se vinculan de forma sistemática el problema de investigación, los objetivos, la hipótesis y las variables del estudio. Esta herramienta facilita la verificación de la coherencia interna de la tesis antes de avanzar con el desarrollo metodológico.

\newcolumntype{Y}{>{\raggedright\arraybackslash}X}

\begin{table}[H]
\centering
\caption{Matriz de Consistencia}
\begin{tabularx}{\textwidth}{|Y|Y|Y|Y|}
\hline
\textbf{Problema} & \textbf{Objetivo} & \textbf{Hipótesis} & \textbf{Variables} \\
\hline
\textbf{General:} ¿Qué patrones lingüísticos diferencian las letras de canciones en español premiadas por la RIAA de aquellas que no han sido premiadas durante el período 1990--2022, y cómo pueden ser identificados mediante un modelo de clasificación supervisado?

\vspace{0.3em}
\textbf{Específicos:}
\begin{enumerate}
  \item ¿Cómo construir un corpus balanceado?
  \item ¿Qué técnicas PLN son adecuadas?
  \item ¿Qué modelos clasifican mejor?
  \item ¿Qué características son más relevantes?
  \item ¿Cómo representar visualmente los resultados?
\end{enumerate}
&

\textbf{General:} Desarrollar un modelo de clasificación supervisado que permita identificar y analizar patrones lingüísticos característicos de las letras de canciones en español premiadas y no premiadas por la RIAA.

\vspace{0.3em}
\textbf{Específicos:}
\begin{enumerate}
  \item Recolectar un corpus balanceado
  \item Preprocesar con técnicas de PLN
  \item Entrenar y comparar modelos
  \item Identificar características lingüísticas
  \item Visualizar patrones relevantes
\end{enumerate}
&

\textbf{General:} Existen patrones lingüísticos diferenciables entre letras premiadas y no premiadas que pueden ser detectados por un modelo con precisión significativa.

\vspace{0.3em}
\textbf{Específicas:}
\begin{itemize}
  \item El contenido lingüístico varía entre clases
  \item Algunos modelos son más precisos
  \item Las visualizaciones ayudan a interpretar resultados
\end{itemize}
&

\textbf{Independiente:} Contenido lingüístico de las letras

\vspace{0.3em}
\textbf{Dependiente:} Éxito musical (premiación)

\vspace{0.3em}
\textbf{Auxiliares:} Desempeño del modelo, visualización e interpretabilidad
\\
\hline
\end{tabularx}
\label{tab:matriz_consistencia}
\end{table}